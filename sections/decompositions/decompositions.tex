% !TEX root = ../../commutative_algebra.tex

\newpage
\section{Primary Decompositions} 

\subsection{Primary Submodules \& Ideals}

\begin{dfn}[Primary Submodule]
Let $N \leq M$ be $R$-modules. We say that $N$ is a primary submodule of $M$ if and only if for every element $a \in R$, multiplication by $a$ on quotient $M/N$ is injective or nilpotent. Equivalently, if $rx \in N$ for some $x \in M$, then either $x \in N$ or $r^n: M \to N$. 
\end{dfn}

Observe that $N \leq M$ is primary if and only if $0 \leq M/N$ is primary. Furthermore, $I \leq R$ is a primary ideal, i.e. a prime submodule of a ring if and only if $ab \in I$ for two elements $a,b \in R$ then either $b \in I$ or $a^n \in I$ for some $n \geq 0$. Consequently for any prime ideal, the power $P^n$ is a prime ideal. To see this, suppose $ab \in P^n$. Then $ab \in P$ so that $a \in p$ or $b \in P$. But then $a^n \in P^n$ or $b^n \in P^n$. 

Now suppose that $N \leq M$ is primary. Then $\sqrt{\ann_R(M/N)}$ is a prime ideal.

\begin{prop}
If $N,M$ are $R$-modules with $N$ a primary submodule of $M$, then
\[
\sqrt{\ann_R(M/N)}=\sqrt{\{r \in R \;|\; rM \subseteq N\}}=\{r \in R\;|\; r^nM \subseteq N\}
\]
\end{prop}

\noindent Proof: Suppose $ab \in \sqrt{\ann_R(M/N)}$ with $a \notin \sqrt{\ann_R(M/N)}$. Then for some $n$, $(ab)^n \in \ann_R(M/N)$ for all $k$. So multiplication by $a$ is not nilpotent on $M/N$. So this map must be injective since $N \leq M$ is primary. Then we have
\[
0=a^n(b^n(M/N))
\]
so that $b^n(M/N)=0$ so that $b \in \sqrt{\ann_R(M/N)}$. \qed \\

Note that if $N \leq M$ is primary by the preceding result, if $P= \sqrt{\ann_R(M/N)}$, we say that $N$ is $P$-primary or that $P$ belongs to $N$. If $I \leq R$ is primary, $\sqrt{\ann_R(R/I)}=\sqrt{I}$ is a prime ideal by the preceding remark. Furthermore, primary ideals have prime radicals but the converse is false.

\begin{ex}
Let $R=k[x,y]$, where $k$ is a field and let $I=(x^2,xy)$. Then using Krull's Intersection Theorem (for the first equality)
\[
\sqrt{I}= \bigcap_{p \in V(I)} p=\bigcap_{x^2,xy \in p} p = \bigcap_{x \in p} p=(x)
\]
is a prime ideal (note the middle equality follows from the fact that if $x^2 \in p$, then $x \in p$ so that $xy \in p$ automatically). It only remains to show that this is not primary. Observe $xy \in I$ and $x \notin I$. But now power of $y \in I$. So that $I$ is not primary. 
\end{ex}

\begin{prop}
If $\sqrt{I}$ is a maximal ideal, then $I$ is primary. In fact, if $\fm=\sqrt{I}$ then $I$ is $\fm$-primary. 
\end{prop}

\noindent Proof: Suppose $ab \in I$ and $b^n \notin I$ for all $n \geq 0$. We want to show that $a \in I$. Since $b^n \notin I$, then $b \notin \sqrt{I}=\fm$. So the ideal $(I,b)$ is not contained in $\fm$. Then it must be that $(I,b)=R$ so $1 \in (I,b)$; that is, $1=x+rb$ for some $x \in I$ and $r \in R$. But then $1=x+rb$ so that $a=ax+rab$. But $ax \in I$ and $rab \in I$ so that $a \in I$. \qed \\

\begin{ex}
Let $R=k[x,y]$, where $k$ is a field and let $I=(x^3,x^2y^4,y^5)$ is primary to a maximal ideal generated by $(x,y)$ since $P \supseteq I$ contains both $x,y$ so that by Krull's Intersection Theorem, $\sqrt{I}=(x,y)$, which is prime. 
\end{ex}

\begin{ex}
In $\Z$, an ideal $I$ is primary if and only if $I=(0)$ or $I=(p^n)$ for a prime $P$ and some $n \geq 0$.
\end{ex}

Consequently by the Fundamental Theorem of Arithmetic, every ideal in $\Z$ can be written uniquely as an intersection of primary ideals with distinct radicals: if $(m) \subset \Z$, where $p_1^{a_1}p_2^{a_2}\cdots p_r^{a_r}$, where the $p_i$ are distinct prime integers, then $(m)=(p_1^{a_1} \cap \cdots (p_r^{a_r})$ and $\sqrt{(p_i^{a_i})}=(p_i)$. It is our goal to show that this is the case for any noetherian ring. That is, given any submodule $N$ of a noetherian module $M$ has a primary decomposition
\[
N= Q_1 \cap \cdots Q_n
\]
where each $Q_i$ is primary. The $Q_i$ are not unique but minimal primary decompositions are unique and prime ideals belonging to them are. 

\subsection{Primary Decomposition}

\begin{dfn}[Irreducible Submodule]
We say that a submodule $N \leq M$ is irreducible if $N$ can be written as
\[
N=N_1 \cap N_2
\]
for $N_i \leq M$, then $N_1=N$ and $N_2=N$.
\end{dfn}

It is our goal to use this to show the existence of primary decompositions. But we do this in an easier manner by showing existence of irreducible decompositions, which we will show are primary. We then worry about the uniqueness. 

\begin{lem}
Assume that $M$ is a noetherian module. Then irreducible submodules are primary.
\end{lem}

\noindent Proof: Let $N \leq M$ be irreducible. Let $r \in R$ and assume that multiplication by $r$ is not nilpotent on $M/N$. Since $M$ is noetherian, $M/N$ is noetherian. We have a chain
\[
\ker r \leq \ker r^2 \leq \ker r^3 \leq \cdots \leq \ker r^n \leq \cdots
\]
This must stabilizer as $M/N$ is noetherian, say at $n$. Then by Fitting's Lemma, $\ker r^n \cap \im r^n=0$ in $M/N$ as $r^nx=0$ and $x=r^ny$. Then $r^{2n}y=0$ so that $r^ny=0$ by stabilization showing that $x=0$. Since $N$ is irreducible in $M$, 0 is irreducible in $M/N$ (by the Correspondence Theorem) showing $\ker r^n=0$ or $\im r^n=0$. But then $\ker r^n=0$ showing that multiplication by $r$ is injective via the chain of kernels. \qed \\

\begin{thmm}[Noether, 1921]
Let $M$ be a noetherian module and $N$ a submodule, then we can write
\[
N= Q_1 \cap \cdots \cap Q_n
\]
of irreducible submodule, hence primary submodules. 
\end{thmm}

\noindent Proof: Let $\Gamma=\{ N \leq M \;|\; N \text{ has no such decomposition}\}$. We want to show that $\Gamma=\emptyset$. Suppose that $\Gamma \neq \emptyset$. By ``noetherianness", $\Gamma$ has a maximal element, say $N$. It must be that $N$ is reducible otherwise it is its own decomposition, so that $N \notin \Gamma$. So $N=N_1 \cap N_2$ with $N_1,N_2 \supsetneq N$. By maximality, these have irreducible decompositions so that $N$ has an irreducible decomposition, a contradiction. Therefore, $\Gamma=\emptyset$. \qed \\

\begin{cor}
Primary decompositions exist for submodules of noetherian modules.
\end{cor} 

Now we need to discuss what we mean by saying we can make this decomposition unique as aforementioned. 

\begin{dfn}[Irredundant]
Let $N=Q_1 \cap \cdots \cap Q_n$ be a primary decomposition. Let $P_i \defeq \sqrt{\ann_R(M/Q_i)}$ be the set of prime ideals belonging to $Q_i$ (note that if $N=I$ is an ideal, the $Q_i$ are prime ideals so that $P_i=\sqrt{Q_i}$). We say that such a decomposition is irredundant or reduced if no $Q_i$ can be removed from the intersection with equality still holding and that all the $P_i$ need be distinct. 
\end{dfn}

It is clear that such a decomposition satisfying the first condition is possible: if you are given a primary decomposition and one of the $Q_i$ is not needed for equality in the intersection, simply leave it out. We need to demonstrate that the second condition can be made to hold. 

\begin{lem}
If $Q_1,\cdots,Q_n$ are $p$-primary, then so too is $Q_1 \cap \cdots \cap Q_k$ (so if several of the $Q_i$ have the same $P_i$, simply ``clump" them together to form the ``smallest" $Q_i$).
\end{lem}

\noindent Proof: We need to show
\begin{enumerate}[(i)]
\item the intersection $Q_1 \cap \cdots \cap Q_k$ is primary
\item $P=\sqrt{\ann_R(M/Q_1 \cap \cdots \cap Q_k)}$
\end{enumerate}



To prove (ii), observe $\ann_R(M/Q_1 \cap \cdots \cap Q_k)=\cap_{i=1}^k \ann_R(M/Q_i)$ (``knocking" $M$ into all the $Q_i$'s must ``knock" $M$ into each one). Since $\sqrt{I \cap J}=\sqrt{I} \cap \sqrt{J}$, we have
\[
\sqrt{\ann_R(M/Q_1 \cap \cdots \cap Q_k)}= \cap_{i=1}^k \sqrt{\ann_R(M/Q_i)}= \cap_{i=1}^k p=p
\]

To prove (i), let $r \in R$. Assume that multiplication by $r$ is not nilpotent on $M/Q_1 \cap \cdots \cap Q_k$. So $r \notin \sqrt{\ann_R(M/Q_1 \cap \cdots \cap Q_k)}=p=\sqrt{\ann_R(M/Q_i)}$ for $i=1,2,\cdots,k$. So multiplication by $r$ is not nilpotent on any $M/Q_i$. Therefore, multiplication by $r$ is injective on each $M/Q_i$ so that multiplication is injective on $M/Q_1 \cap \cdots \cap Q_k$. \qed \\

\begin{cor}
Irredundant primary decompositions exist for noetherian modules.
\end{cor}

\begin{thmm}
Let $N=Q_1 \cap \cdots \cap Q_n$ be an irredundant primary decomposition for an $R$-submodule $N$, where $R$ is noetherian. Let $P_i$ be the prime ideal belonging to $Q_i$:
\begin{enumerate}[(i)]
\item the set $\{p_1,\cdots,p_n\}$ is determined uniquely by $N$
\item the $Q_i$'s corresponding to isolated primes $P_i$ (does not contain the other $P_j$'s) are also determined by $N$
\end{enumerate}
\end{thmm}

\begin{dfn}[Colon Ideals]
If $N \leq M$ is a submodule, we define
\[
(N:_R M) \defeq \{r \in R \;|\; rM \subseteq N\}
\]
If $x \in M$, we define
\[
(N:_R x) \defeq \{r \in R \;|\; rx \in N\}
\]
\end{dfn}

\begin{lem}
Let $Q \leq M$ be a $p$-primary submodule, i.e. 
\[
\sqrt{\ann_R(M/Q)}=p
\]
Let $x \in M$. If $x \in Q$, then $\sqrt{(Q:_R x)}=R$ and if $x \notin Q$ then $\sqrt{(Q:_R x)}=p$. This shows we can write $(Q:_R x)$ to obtain $p$ instead of using $\ann_R(M/Q)$.
\end{lem}

\noindent Proof: If $x \in Q$, then $(Q:_R x)=R$. To prove the second equality, let $r \in \sqrt{(Q:_Rx)}$ then $r^nx \in Q$ for some $n$. So multiplication by $r^n: M/Q \to M/Q$ is not injective so that it must be nilpotent. So $r^m \in \ann_R(M/Q)$ for some $m$. Hence, $r \in p$. For the other inclusion, let $r \in p$ so that $r^n$ annihilates $M/Q$ for some $n$. In particular, $r^nx \in Q$. \qed \\

\begin{thmm}[First Uniqueness Theorem]
Let $R$ be a noetherian ring and $N \leq M$ be $R$-modules. Suppose that $N=Q_1 \cap \cdots \cap Q_s$ is an irredundant primary decomposition of $N$ in $M$. Let $p_i=\sqrt{\ann_R(M/Q_i)}$ be the primes belonging to the $Q_i$. For any $p \in \spec R$, we have $p \in \{p_i\}_{i=1}^s$ if and only if $p=(N:_R x)$ for some $x \in M \setminus N$. Hence, the $p_i$ are uniquely determined by $N$ and do not depend on the choice $\{Q_i\}_{i=1}^s$. These primes are called the associated primes of $N$ and we write $\{p_i\}_{i=1}^s= \ass_R(M/N)$. In the special cases of $I \leq R$, we write $\ass_R(I) \defeq \ass_R(R/I)$ and if $I=(0)$, we write $\ass_R(R)\defeq \ass_R((0))$.
\end{thmm}

\noindent Proof: To prove the reverse direction, suppose that $p=(N:_R x) \neq R$. Then we have
\[
\begin{split}
p=(N:x)&= (Q_1 \cap \cdots \cap Q_s: x) \\
&=(Q_1 :x) \cap \cdots \cap (Q_s: x) \\
&\supseteq (Q_1:x)(Q_2:x)\cdots(Q_s:x)
\end{split}
\]
But as $p$ is prime, we must have $p \supseteq (Q_i:x)$ for some $i$. As $p \neq R$, we must have $x \notin Q_i$. Furthermore as $p$ is prime, $p \supseteq \sqrt{(Q_i:x)}=p_i$ by the preceding lemma. On the other hand, $p=\cap (Q_j:x) \subseteq (Q_i:x) \subseteq \sqrt{(Q_i:x)}$. Therefore, $p=\sqrt{(Q_i:x)}=p_i$. 

For the forward direction, without loss of generality, we show that $p_1=(N:x)$ for some $x \in M \setminus N$. As the primary decomposition is irredundant, $N \subsetneq Q_2 \cap \cdots \cap Q_s$. Then the set $\Gamma=\{I \subseteq R \;|\; I=(N:x), x \in (Q_2 \cap \cdots \cap Q_s) \setminus N\}$ is nonempty. As $R$ is noetherian, we can find a maximal element, say $(N:x)$. 

We claim that $\ann_R(M/Q_1) \subseteq (N:x) \subseteq p_1=\sqrt{\ann_R(M/Q_1)}$. To see that, $\ann_R(M/Q_1) \subseteq (N:x)$, suppose that $r \in \ann_R(M/Q_1)$. Then $rx \in Q_1$ and we also have $x \in Q_2 \cap \cdots \cap Q_s$. Then it must be that $rx \in Q_1 \cap \cdots \cap Q_s=N$. To see that $(N:x) \subseteq p_1$, suppose that $s \in (N:x)$. Then $sx \in N \subseteq Q_1$. However, $x \notin Q_1$ so that $s$ is not injective on $M/Q_1$ so that $s$ is nilpotent. Therefore, $s \in \sqrt{\ann_R(M/Q_1)}=p_1$. 

The fact that we have $\ann_R(M/Q_1) \subseteq (N:x) \subseteq p_1=\sqrt{\ann_R(M/Q_1)}$ implies that $p_1=\sqrt{(N:x)}$. We want to show $p_1=(N_1:x)$; that is, we want to show that the radical is redundant. It is enough to show that $(N:x)$ is prime. Let $a,b \in R$ such that $ab \in (N:x)$ and $a \notin (N:x)$. Then $abx \in N$ so that $b \in (N:ax)$. We show that $(N:ax)=(N:x)$ via maximality. Note that $(N:x) \subseteq (N:ax)$. To it is sufficient to show $(N:ax) \in \Gamma$. For this to be so, we need $ax \in (Q_2 \cap \cdots \cap Q_s) \setminus N$. But $x \in Q_2 \cap \cdots \cap Q_s$ so that $ax \in Q_2 \cap \cdots \cap Q_s$ also. But $a \notin (N:x)$ so that $ax \notin N$. But then $(N:ax) \in \Gamma$ so that $(N:ax)=(N:x)$ by maximality. But then $b \in (N:x)$, as desired. \qed \\

\begin{rem}
Note that if we say that $p$ is an associated prime to the $R$-module $M$ if $N=0$ in the First Uniqueness Theorem. This is equivalent to saying that there is a $m \in M$ such that $p=\ann_R(m)$. 
\end{rem}

\begin{ex}
Let $R=k[x,y]$, where $k$ is a field, and $I=(x^2,xy)$. Notice that $I=(x) \cap (x^2,y)$ and $I=(x) \cap (x^2,xy,y^2)$. These are both irredundant primary decompositions. Then $(x)$ is prime showing that it is primary. Furthermore, $\sqrt{(x^2,y)}=\sqrt{(x^2,xy,y^2)}=(x,y)$ is maximal, so it is primary also. But then $\ass_R(R/I)=\ass_R(I)=\{(x),(x,y)\}$. 

We should be able to write these in the form $(I:f)$ for some $f \notin I$. Notice that $xy \in I$ so that $x \in (I:y)$. In fact, if $g \in (I:y)$, then $gy \in I \subseteq (x)$ so that $g \in (x)$. Hence, $(x)=(I:y)$. For the other, notice that $x^2,xy \in I$ so that $(x,y) \subseteq (I:x)$. But as $(x,y)$ is maximal and $(I:x) \neq R$ (as $x \notin I$), we have $(x,y)=(I:x)$. 
\end{ex}

\begin{dfn}[Isolated/Embedded Primary Component]
Suppose that $N=Q_1 \cap \cdots \cap Q_s$ is an irredundant primary decomposition and $p_i=\sqrt{\ann_R(M/Q_i)}$. If $p_i$ is a minimal element of $\ass_R(M/N)$, we say that $Q_i$ is an isolated primary component. Otherwise, we say that $Q_i$ is an embedded primary component
\end{dfn}

\begin{ex}
Given the notation of the previous example, we have $\minass(R/I)=\{(x)\}$ is isolated and $(x,y)$ is embedded. 
\end{ex}

\begin{thmm}[Second Uniqueness Theorem]
Let $N \leq M$ by $R$-modules in a noetherian ring and let $N=Q_1 \cap \cdots \cap Q_s$ be an irredundant primary decomposition for $N$ in $M$. Suppose that $Q_i$ is isolated (so $p_i$ does not contain any $p_j$), then
\[
Q_i=\left\{x \in M \;|\; \frac{x}{1} \in N_{p_i} \right\}= \varphi^{-1}(N_{p_i})
\]
where $\varphi: M \to M_{p_i}$. In particular, $Q_i$ is in every primary decomposition for $N$. 
\end{thmm}

\noindent Proof: For notational ease, let $p=p_i$. Suppose that $x \in M$. Then $\frac{x}{1} \in N_p$ and $\frac{x}{1}=\frac{n}{s}$ for some $n \in N, s \in R \setminus p$. So there is a $t \notin p$ with $t(sx-n)=0$, i.e. $tsx=tn \in N$. Set $r=ts$ so that $rx \in N \subseteq Q$. Notice that $r \notin p$ and $p=\sqrt{\ann_R(M/Q)}$ so multiplication by $r$ is not nilpotent on $M/Q$. It follows that multiplication by $r$ must be injective. But then $x \in Q$. 

Now assume that $x \in Q$. Since $p$ is minimal in $\ass_R(M/N)$, $\cap_{j \neq i} p_j \not\subseteq p$ (if it were, then $p_1p_2\cdots \hat{p}_i \cdots p_n \subseteq p_1$ so that $p_j \subseteq p_i$, a contradiction). So choose $a \in \cap_{j \neq i} p_j \setminus p$. Then $a \in \sqrt{\ann_R(M/Q_j)}$ for all $j \neq i$. But then multiplication by $a$ is nilpotent on $M/Q_j$ for all $j \neq i$. Choosing $k$ sufficiently large such that $a^kM \subseteq \cap_{j \neq i} Q_j$, then $a^kx \in \cap_{j \neq i} Q_j$ and also $Q_i$. But then $a^kx \in N$ so that $\frac{x}{1}=\frac{a^kx}{a^k} \in N_p$ since $a^k \notin p$. To see that $Q_i$ appears in every primary decomposition for $N$, note that we have $Q_i=\varphi^{-1}(N_{p_i})$ so that $Q_i$ is uniquely determined by $N$. By the First Uniqueness Theorem, we know that $p_i$ depends only on $N$. But $Q_i$ is determined by $p_i$ so that $Q_i$ is determined by $N$. \qed \\

\subsection{Applications of Associated Primes}

\begin{dfn}[Zerodivisor]
We say that $r \in R$ is a zero divisor on $M$ if there is a $x \in M\setminus \{0\}$ such that $rx=0$. We denote by $Z_R(M)$ the set of zero divisors in $M$. 
\end{dfn}

\begin{prop}
Let $R$ be noetherian and $N \leq M$ be finitely generated. Then
\[
Z(M/N)=\bigcup_{p \in \ass_R(M/N)} p
\]
\end{prop}

\noindent Proof: If $p \in \ass(M/N)$, then $p=(N:_R x)$ for some $x \in M \setminus N$. So every $r \in p$ satisfies $r \ov{x}=0$ in $M/N$ and so every element of $p$ is a zero divisor. Now suppose that $r \ov{x}=0$ for some $\ov{x} \in M/N$ with $\ov{x} \neq 0$. Let $N=Q_1 \cap \cdots \cap Q_s$ be an irreducible primary decomposition for $N$ in $M$. Since $x \notin N$ then $x \notin Q_i$ for some $i$. However, $rx \in Q_i$ so that multiplication by $r$ is not injective on $M/Q_i$. Hence, multiplication by $r$ must be nilpotent. This shows that $r \in \sqrt{\ann_R(M/Q_i)}=p_i$. \qed \\

\begin{ex}
Let $R=k[x,y]/(x^2,y)$, where $k$ is a field. Then $Z(R)=\cup_{p \in \ass(R)} p$. In $k[x,y]$, $(x^2,xy)=(x) \cap (x^2,y)$ so that in $R$, $(0)=(\ov{x}) \cap (\ov{x}^2,\ov{x}\ov{y})$. So $\ass R=\{(\ov{x}),(\ov{x},\ov{y})\}$ and $Z(R)=(\ov{x},\ov{y})$. But then $R$ is local (recalling that a ring is local if and only if the non units form an ideal).
\end{ex}

\begin{prop}
Let $R$ be noetherian and $I \leq R$. Then
\[
\sqrt{I}= \bigcap_{p \in \ass_R(R/I)} p
\]
\end{prop}

\noindent Proof: If $I=Q_1 \cap \cdots \cap Q_s$, then $\sqrt{I}=\sqrt{\cap Q_i}=\cap \sqrt{Q_i}=\cap p$. \qed \\

\begin{dfn}[$n$th-symbolic power]
Let $p \in \spec R$. The $n$th symbolic power of $p$ is 
\[
p^{(n)}=p^nR_p\cap R= \varphi^{-1}(p^nR_p)
\]
where $\varphi: R \to R_p$.
\end{dfn}

\begin{rem} We have the following properties for symbolic powers: 
\begin{enumerate}[(i)]
\item $p^n \subseteq p^{(n)}$
\item If $\varphi: A \to B$ and $I \subseteq B$ is primary, then $\varphi^{-1}(I)$ is primary. 
\item $p^nR_p$ is a power of the maximal ideal of $R_p$ so that it is primary. 
\item $p^{(n)}$ is $p$-primary. [It is enough to show that $p^{(n)} \subseteq p$. If $x \in p^{(n)}$ then $\frac{x}{1} \in p^n R_p \subseteq pR_p$ so that $\frac{x}{1}=\frac{r}{s}$ for some $r \in p, s\notin p$. But then there is a $t \notin p$ with $stx=tr \in p$. However, $s,t \notin p$ so that $x \in p$.]
\item $p^{(n)}$ is the $p$-primary component of $p^n \subseteq R$. Because $p$ is isolated, any other associated prime of $p^n$ must contain $p$.
\end{enumerate}
\end{rem}

\begin{lem}[Prime Avoidance]
Let $R$ be a ring and $I,P_1,P_2,\cdots,P_n$ be ideals of $R$ with $n-2$ of the $P_i$ primes. If $I \subset \cup_{i=1}^n P_i$, then $I$ is contained in $P_i$ for some $i=1,2,\cdots,n$. 
\end{lem}

\noindent Proof: We proceed by induction on $n$. For the case of $n=1$, there is nothing to show. If $n=2$, then we have $p_1 \cup p_2 \supseteq I$. Assume to the contrary that $I \not\subseteq p_1,p_2$. Choose $x \in I$ such that $x \notin I \setminus p_1$ and $y \in I \setminus p_2$. Now we have $x+y \in I$ but $x+y \notin p_1,p_2$. But then $y=(x+y)-x \in p_2$, a contradiction. Notice this did not even use the fact that $p_i$ was prime.

Now assume that $n \geq 3$ and $I \subseteq \cup p_i$ but $I \not\subseteq p_j$ for $j=1,2,\cdots,n$. If $I \subseteq p_1 \cup \cdots \cup \hat{p}_k \cup \cdots \cup p_n$, then by induction $I$ must be in one of them and we are done. So if $I \not\subseteq \cup p_j$ with $j \neq k$ for all $k$. That is, 
\[
I \not\subseteq \cup_{j \neq k} p_j
\]
Then at least one of the $p_j$'s are prime, say $p_1$. For each $k=1,2,\cdots,n$, choose $a_k \in I \setminus \cup_{j \neq i} p_j$. Note that $a_k \in p_k$ for each $k$. But then $y \defeq a_1+(a_2a_3 \cdots a_n) \in I$. If $y \in p_1$, then as $a_1 \in p_1$, we have $a_2 \cdots a_n \in p_1$ so that $a_k \in p_1$ for some $k \neq 1$, a contradiction. Then it must be that $y \notin p_1$ so that $y=p_1$ for some $j \neq 1$ so that $a_1 \in p_j$, a contradiction. \qed \\

Note that this Lemma gets its name from its contrapositive: if $I$ is not in any $p_j$, then $I$ is not contained in the union of the $p_j$'s. So there is some element of $I$ ``avoiding" all of the $p_j$'s.

\begin{ex}
If $(R,\fm)$ is a noetherian local ring with maximal ideal $\fm$ and $\fm$ is not an associated prime of $R$, then $\fm \not\subseteq p$ for all $p \in \ass(R)$ (by maximality). Then there is a $r \in M$ such that $r$ is not an element of any associated primes. But then $r$ is a nonunit and nonzerodivisor. 
\end{ex}

\begin{rem}
If a ring contains an infinite field, then more than one of the $p_i$ have to be prime. This is observed using the fact that there is no vector space with dimension greater than 1 is an infinite field is a union of finitely many subfields.
\end{rem}

\subsection{Ideal/Variety Correspondence}

\begin{thmm}[Hilbert's Basis Theorem]
If $R$ is noetherian then $R[x]$ is noetherian.
\end{thmm}

\noindent Proof (Sketch): Let $I$ be an ideal of $R[x]$. We will show that $I$ is finitely generated. Set $I_d$ to be the set of leading coefficients of polynomials $f(x) \in I$ with degree $d$ along with 0. Then $I_d$ is an ideal of $R$ and we have a chain of ideals as $I_d \subseteq I_{d+1}$. As $R$ is noetherian, this must stabilize at some $n$, say $N$. Then for $d=0,\cdots,N$, write $I_d=(c_{d,1},\cdots,c_{d,n_d})$. Then the $c_{i,j}$ are the leading coefficients of some $f_{i,j} \in I$ so that $I=(f_{i,j})_{N,n_i}$ by choosing elements of minimal degree of the left hand side but not the right, a contradiction. \qed \\

\begin{cor}
If $R$ is noetherian, then so is any finitely generated $R$-algebra. 
\end{cor}

\noindent Proof: If $S=R[u_1,\cdots,u_n]$, where $u_i \in S$, then $S \cong R[x_1,\cdots,x_n]/\ker \phi$, where $\phi: R[x_1,\cdots,x_n] \to S$ given by $x_i \mapsto u_i$. But the quotient of a noetherian module is noetherian. \qed \\

\begin{rem}
Using roughly the same argument as in the Hilbert Basis Theorem, one can show that if $R$ is noetherian then $R[[x]]$ is noetherian except one uses $I$ to be coefficients of least degree. 
\end{rem}

\subsection{Affine Algebraic Varieties}

Throughout this section, let $k$ be a field.

\begin{dfn}[Affine Space]
An affine $n$-space over $k$ is $\A_k^n=k^n$. The elements of an affine space are called points. 
\end{dfn}

\begin{dfn}[Zero Set]
Let $S$ be any set of polynomials in $k[x_1,\cdots,x_n]$. The zero set of $S$ is $Z(S) \defeq \{ p=(a_1,\cdots,a_n) \in \A_k^n \;|\; f(p)=0 \text{ for all }f \in S\}$. Such a set is called an (affine algebraic) variety. 
\end{dfn}

\begin{ex} We can represent common graphs as affine algebraic sets (giving them an algebraic definition): 
\begin{enumerate}[(i)]
\item If $k=\R$ and $S=\{x^2+y^2-1\} \subseteq k[x,y]$. Then $Z(S)$ is the unit circle in $\R^2$. 

\item If $k=\R$ and $S=\{(x-y)(y-x^2)\}$, then $Z(S)$ is the union of the parabola $y=x^2$ and the line $y=x$. 

\item If $k=\R$ and $S=\{x,y(y-1)\}$, then $Z(S)$ consists of the points $(0,0)$ and $(0,1)$. 

\item If $k=\R$ then $Z(y-2x,z-3x,3y-2z)$ is the line spanned by the vector $\langle 1,2,3 \rangle$. 

\end{enumerate}
\end{ex}

\begin{rem}
Observe that if $S \subseteq S'$ then $Z(S) \supseteq Z(S')$ as anything that kills everything in $S'$ certainly kills everything in $S$, so zero sets are inclusion reversing. Furthermore, we have $Z(S)=Z(\langle S \rangle)$. To see this, note the forward inclusion is immediate as $S \subseteq \langle S \rangle$. To see the reverse inclusion, let $p \in Z(S)$ and $f \in \langle S \rangle$. Then there is a $f_i \in S$ and $g_i \in k[x_1,\cdots,x_n]$ such that $f=\sum g_i f_i$. But then $f(p)=\sum g_i(p)f_i(p)=0$. 
\end{rem}

\begin{rem}
Since $Z(S)$ depends only on ideals by the Hilbert Basis Theorem for any $S\subseteq k[x_1,\cdots,x_n]$, there is a finite generating set for the ideal $\langle S \rangle$, say $\langle f_1,\cdots,f_m \rangle$. Then $Z(S)=Z(\langle S \rangle)=Z(f_1,\cdots,f_m)$. So an affinity algebraic varieties is the common zero set of a finite number of polynomials. Notice that we can actually choose the generators $f_i$ to be in $S$. To see this, we use the noetherian property. Let $\Gamma$ be the set of ideals in $k[x_1,\cdots,x_n]$ that are finitely generated by elements of $S$. Note that we require $S$ to be nonempty. Then $\Gamma$ has a maximal element, say $J$. We want $J=\langle S \rangle$. Note that if $J \subseteq \langle S \rangle$ and $J \subsetneq \langle S \rangle$, we could add more generators, contradicting the maximality of $J$.
\end{rem}

\begin{dfn}[Hypersurface]
A hypersurface is a zero set in $\A_k^n$ fo a single polynomial: $Z(f)$. 
\end{dfn}

Notice that if $\langle S \rangle = \langle f_1,\cdots,f_m \rangle$, then $Z(S)=Z(f_1,\cdots,f_m)=Z(f_1) \cap \cdots \cap Z(f_n)$. So any affine algebraic variety is an intersection of finitely many hypersurfaces. 

\begin{ex}[Macaulay's Curve]
Let $C \subseteq \A_k^4$ be the locus of points of the form $(s^4,s^3t,st^3,t^4)$ for $s,t \in k$, i.e. the image of the map $\phi: \A_k^2 \to \A_k^4$ given by $(s,t) \mapsto (s^4,s^3t,st^3,t^4)$. Then $C$ is a 2-dimensional variety in $\A_k^4$. Using $(x,y,z,w)$ for $\A_k^4$, then $C$ is cut out by the polynomials $xw-yz, y^4-x^3w,z^4-xw^3$. One can show that these generate $C$. Notice that $\codim C=\dim \A_k^4-\dim C=4-2=2$. Does there exist two polynomials alone cutting out $C$? If $\ch k=p$, then the answer is yes (this is due to Hartshorne in the 1960s). However, if $\ch k=0$, then this is an open problem. 
\end{ex} 

\begin{rem}
We have $Z( \cup S)= \cap Z(S)$ as any points killing any one of the $S$'s must kill them all. Furthermore, we have $Z(I_1) \cup \cdots \cup Z(I_m)=Z(I_1 \cdots I_m)$. So see this, if $p$ is an elements of the left side, then $p \in Z(I_j)$ for some $j$. 
\end{rem}

The preceding remarks, as before, show that affine algebraic sets form the closed sets for a topology, called the Zariski topology. 

\begin{dfn}[Zariski Topology]
The topological space over $\A_k^n$ whose closed sets are the affine algebraic sets is called the Zariski topology. 
\end{dfn}

The Zariski topology is $T_1$ (that is, points are closed). To see this, let $p=(a_1,\cdots,a_n) \in \A_k^n$ and set $\fm_p-(x_1-a_1,\cdots,x_n-a_n) \in k[x_1,\cdots,x_n]$. Then we have $Z(\fm_p)=\{p\}$. Notice that this was \emph{not} the case for the Zariski topology on $\spec R$. However, the Zariski topology is still not Hausdorff if $k$ is infinite; that is, the Zariski topology is not $T_2$. 

If $Z(I) \neq \emptyset$ for all ideals $I \subseteq k[x_1,\cdots,x_n]$, then the Zariski topology is pseudo-compact (quasi-compact).

\begin{rem}
Quasi-Compact means that any open covering has a finite subcovering. Often, one meets this as the definition of compact. However, nearly all spaces one typically deals with in Topology, all those in Analysis, et cetera are Hausdorff. However, the only topology an algebraic geometer would care to work with - the Zariski topology - is not. There needs to be a distinct. Quasi-compact is ``compact" as one typically knows it while ``compact" is reserved for compact and Hausdorff. This tedious (though necessary) distinction exists only for algebraic geometers. 
\end{rem}

However, $Z(I) \neq \emptyset$ does not have to hold. For example, take $k=\R$ and observe $Z(x^2+1) \subseteq \A_k^2$ is empty. The moral of this story is this: algebraic geometry works best (perhaps only at all) over an algebraically closed field. It is also interesting to note that the Zariski topological space on an affine $\R$-space over the reals is coarser than the Euclidean topology. Now we go the other direction:

\begin{dfn}[Vanishing Ideal]
For $Y \subseteq \A_k^n$, the vanishing ideal of $Y$ is
\[
I(Y) \defeq \{ f \in k[x_1,\cdots,x_n] \;|\; f(p)=0 \text{ for all } p \in I \}
\]
\end{dfn}

Observe that if $Y \subseteq Y'$ then $I(Y) \supseteq I(Y')$. Furthermore, we know that $Z(I(Y)) \supseteq Y$. In fact, $Z(I(Y))= \ov{Y} \supseteq Y$. 

\begin{lem}
$Z(I(Y))=\ov{Y}$
\end{lem}

\noindent Proof: We know that $Z(I(Y))$ contains $Y$ and is closed. So assume that $Y$ is contained in some closed set $Z(S)$. We want to show that $Z(I(Y)) \subseteq Z(S)$. Let $p \in Z(I(Y))$. Then we have $f(p)=0$ for all $f \in I(Y)$. Let $g \in S$ so that $g \in S \subseteq I(Z(S)) \subseteq I(Y)$, as $Y \subseteq Z(S)$. Therefore, $g(p)=0$ so that $p \in Z(S)$. \qed \\

Then we know that $I(Y)=I(\ov{Y})$. Furthermore, we have $I(\cup Y_\alpha)=\cap_\alpha I(Y_\alpha)$. Finally, we have $I(Y_1 \cap Y_2) \supseteq I(Y_1)+I(Y_2)$ but equality need not hold. To see this, let $f^n \in I(Y)$ so that $f^n(p)=0$ for all $p \in Y$. But we are in a field so that $f(p)=0$ so that $f \in I(Y)$. Now $I(-)$ is always a radical ideal. The sum of radical ideals need not be radical. 

\begin{ex}
Let $Y_1=Z(y)$ and $Y_2=Z(y-x^2)$. Then $I(Y_1)$ is all the multiples of $y$ and $I(Y_2)=(y-x^2)$. We have $I(Y_1)+I(Y_2)=(y,x^2)$. However, we have $Y_1 \cap Y_2=(0,0)$ so that $I(Y_1 \cap Y_2)=(x,y)$. 
\end{ex}

If $I$ is a proper ideal of the polynomial ring $k[x_1,\cdots,x_n]$, then $Z(I) \neq \emptyset$ (this follows from the generalized Fundamental Theorem of Algebra). Finally, we have $I(Z(I))=\sqrt{I}$ for all ideals $I$ of $k[x_1,\cdots,x_n]$. It is our goal to move forwards toward and prove Hilbert's Nullstellensatz: 

\begin{thmm}[Hilbert's Nullstellensatz]
Let $k$ be an algebraically closed field. 
\end{thmm}

There are many good results which follow from this theorem:

\begin{enumerate}[(i)]

\item

\begin{prop}
The maps $I \mapsto Z(I)$ and $Y \mapsto I(Y)$ is a inclusion reversing bijection between the set of radical ideals in $k[x_1,\cdots,x_n]$ and varieties in $\A_k^n$. 
\end{prop}

\noindent Proof: We want to show that $I(Z(I))=I$ if $I$ is radical and $Z(I(Y))=Y$ if $Y$ is a variety. For the first direction, observe that $I=\sqrt{I}$ be the Nullstellensatz. For the reverse direction, if $Y=Z(I)$ for some ideal $I$, we may assume that $I$ is radical as $Z(I)=Z(\sqrt{I})$. But then we have $Z(I(Y))=Z(I(Z(I)))=Z(\sqrt{I})=Y$. \qed \\

Note that this proposition is an example of a Galois correspondence. 

\item

\begin{prop}
A system of polynomial equations $f_1(x_1,\cdots,x_m)=0$, $f_2(x_1,\cdots,x_n)=0$, $\cdots$, $f_m(x_1,\cdots,x_n)=0$ has a simultaneous solution if and only if $I(f_1,\cdots,f_n)$ is a proper ideal of the polynomial ring $k[x_1,\cdots,x_n]$. Equivalently, the above system of polynomial equations has no solutions if and only if $1=\sum_{i=1}^m p_if_i$ for some polynomials $p_1,\cdots,p_m$. 
\end{prop}

\noindent Proof: If $1$ is a linear combination of the $f$'s, then the system cannot have a solution for then the right side of the equality would vanish at that point while the left side - 1 - would not. Conversely, if $Z(f_1,\cdots,f_m)= \emptyset$, then the ideal is not proper by the Nullstellensatz. \qed \\

\begin{dfn}[Coordinate Ring]
For a variety, $X \subseteq \A_k^n$, the coordinate ring of $X$, $k[X]$, is $k[x_1,\cdots,x_n]/I(X)$.
\end{dfn}

Observe that $k[X]$ is a finitely generated $k$-algebra and is reduced since $I(X)$ is a radical ideal. If a $k$-algebra is finitely generated and reduced, then it is an affine $k$-algebra. Conversely, if $k=\ov{k}$, then any affine $k$-algebra is a coordinate ring. If $S$ is an affine $k$-algebra is generated by $u_1,\cdots,u_n$ then we get a surjection $\pi: k[x_1,\cdots,x_n] \to S$ via $x_i \mapsto u_i$. So we have $S \cong k[x_1,\cdots,x_n]/I$, where $I=\ker \pi$ is a radical ideal since $S$ is reduced. So $S$ is in the coordinate ring of $X=Z(I)$. Then we get an equivalence of categories between varieties in affine $n$-space and affine $k$-algebras. 

\item The maximal ideals of $k[x_1,\cdots,x_n]$ are all of the form $\fm_p=(x_1-a_1,\cdots,x_n-a_n)$ for $p=(a_1,\cdots,a_n) \in \A_k^n$. Note that this fails if $k \neq \ov{k}$. As an example, $(x^2+1) \in \R[x]$ is maximal as $\R[x]/x^2+1 \cong \C$, a field. So the map $\A_k^n \to \maxspec k[x_1,\cdots,x_n]$ given by $p \mapsto \fm_p$ is a bijection. 

\item If $X \subseteq \A_k^n$ is a variety, then $X \to \maxspec k[X]$ given by $p \mapsto \ov{\fm}_p$ is a bijection. [$p \in X$ if and only if $I(X) \subseteq I(p)=\fm_p$]

\end{enumerate}

In order to prove the Nullstellensatz, we will need to divert and discuss dimension theory and some advanced field theory. 