% !TEX root = ../../commutative_algebra.tex

\newpage
\section{Gorenstein Rings}

\begin{dfn}[Gorenstein, Old Definition]
Let $(R,\fm,k)$ be a noetherian local ring. We say that $R$ is gorenstein if $R$ is CM and there is a system of parameters $x_1,\cdots,x_d$, where $d=\dim R$, such that they $x_1,\cdots,x_d$ generate an irreducible ideal, i.e. $(x_1,\cdots,x_d) \neq I \cap J$ for two ideals $I,J \neq (x_1,\cdots,x_d)$. 
\end{dfn}

\begin{enumerate}[1.]
\item The definition does not depend on the choice of system of parameters. 
\item In CM local rings, system of parameters are regular sequences. So $R$ is gorenstein if and only if it is CM and $(0)$ is irreducible in $R$ modulo the system of parameters. This means it is essential to understand the 0-dimensional case. 
\end{enumerate}

\begin{ex}
Let $R=k[x]/x^2$ has only 3 ideals $(0) \subsetneq (x) \subsetneq k[x]$. The zero ideal clearly is not the intersection of two nonzero ideals so $R$ is gorenstein. 
\end{ex}

\begin{ex}
Let $R=k[x,y]/(x^2,xy,y^2)$. We have $(0)=(x) \cap (y)$ so that $R$ is not gorenstein. 
\end{ex}

Note that $R$ is gorenstein if $R$ is 0-dimensional and $(0)$ is irreducible. If $\dim R>0$, then $R/(x)$ is 0-dimensional and gorenstein for some regular sequence $\vec{x}$.

\begin{ex}
Regular local rings are gorenstein. Why? We know that if $R$ is regular local then $M$ is generated by a regular sequence. Then the field $R/\fm$ is gorenstein. 
\end{ex}

\begin{ex}
If $R$ is a regular local ring and $f_1,\cdots,f_c$ is a regular sequence then $R/(f_1,\cdots,f_c)$ is a gorenstein ring as we can extend the sequence $f_1,\cdots,f_c,g_1,\cdots,g_d$ so that $R/(f_1,\cdots,g_d)$ is gorenstein by the independence of the system of parameters so the image of $g_1,\cdots,g_d$ in $R/(f_1,\cdots,f_c)$ give the needed system of parameters. 
\end{ex}

\begin{dfn}[Complete Intersection Ring]
A complete intersection ring is a local ring $(R,\fm,k)$ such that the completion $\hat{R}$ is of the form $Q/(f_1,\cdots,f_c)$, where $Q$ is a regular local ring and $f_1,\cdots,f_c$ is a regular sequence in $Q$. 
\end{dfn}

\begin{rem}
By the Cohen Structure Theorem, every complete local ring is a regular local ring modulo some ideal $I$.
\end{rem}

\begin{dfn}[Socle]
The socle of a local ring $(R,\fm,k)$ is written $\soc(R)=\ann_R(\fm)$ or equivalently $(0:_R \fm)$. It is a vector space over $k$ since anything in it is killed by $\fm$. It is finitely generated as $R$ is noetherian so it is a finite dimensional vector space.
\end{dfn}

\begin{dfn}[Type]
The type (or Cohen-Macaulay type) is $r(R)=\defeq \dim_{R/\fm}(\soc R)$.
\end{dfn}

 Recall an extension of modules $N \subseteq M$ is essential if every $L \leq M$ has $N \cap L \neq 0$. 
 
 \begin{lem}
 If $(R,\fm,k)$ is a 0-dimensional local ring then $\soc R$ is an essential ideal of $R$, i.e. $\soc R \subseteq R$ is essential. 
 \end{lem}
 
 \noindent Proof: Let $0 \neq K \leq R$. We know that $\fm^n=0$ for $n$ sufficiently large (since in 0-dimensional rings the radical is nilpotent). So $\fm^nk=0$ for $n$ sufficiently large. Choose $n$ to be minimal, i.e. $\fm^{n-1}k \neq 0$ but $\fm^nk=0$. Then $\fm(\fm^{n-1}k)=0$ so $\fm^{n-1}k \subseteq \soc R$. It is also clearly in $k$ so that it is in $k \cap \soc R$ so the intersection is nontrivial. \qed \\

\begin{prop}
Let $(R,\fm,k)$ be a 0-dimensional local ring. Then $R$ is gorenstein if and only if it is type 1, i.e. $r(R)=1$.
\end{prop}

\noindent Proof: Note that the lemma shows that $\soc R \neq 0$. For the forward direction, suppose $\dim_k(\soc R)>1$ so that $\dim \geq 2$. Find $x,y \in \soc R$ is linearly independent over $k$. Then $(x) \cap (y)=(0)$ since if $ax=by \neq 0$ so that $a,b \notin \fm$ (since if it is in the socle, it is killed by $\fm$). Then we have a linear relation over $R/\fm$ so that $R$ is not gorenstein. 

For the reverse direction, assume that $\soc R$ is 1-dimensional. By the lemma, $I \cap \soc R$ is not empty for all $0 \neq I \lhd R$. But $I \cap \soc R$ is a subspace of $\soc R$ so $I \cap \soc R=\soc R$ for all $I \neq 0$, i.e. every nonzero ideal contains the socle. Then we cannot have $(0)=I \cap J$ since $\soc R$ is in the right hand side so $R$ is gorenstein. \qed \\

\begin{ex}
Let $R=k[x,y]/(x^2,y^4)$ is a complete intersection ring so it is gorenstein. What is the socle? We create a Pascal like triangle
\begin{center}
\begin{tabular}{ccccc}
$1$ & & & & \\
$x$ & $y$ & & &\\
$x^2$ & $xy$ & $y^2$ & & \\
 & & $xy^2$ & $y^3$ \\
  & & & & $y^4$ \\
\end{tabular}
\end{center}
So it is $(xy^3)$ since anything killed by $x,y$ and anything killed by $(x^2,y^4)$ is killed by $x$ and $y^3$ so it is a multiple of $xy^3$. 
\end{ex}

\begin{ex}
Let $R=k[[x,y]]/(x^2,xy)$. Then $\soc R=(x)$ is a 1-dimensional vector space. 
\begin{center}
\begin{tabular}{cccccc}
$1$ & & & & & \\
$x$ & $y$ & & & \\
$x^2$ & $xy$ & $y^2$ & & & \\
 & & & $y^3$ & & \\
 & & & & $y^4$ & \\
 & & & & & $\ddots$
\end{tabular}
\end{center}
where everything from $x^2,xy$ and under is $0$ in the ring. But notice the right diagonal is never zero so that this is not gorenstein. Now $V=(x^2,xy)=V(x)$ so this is geometrically 1-dimensional (the $y$-axis). So the ring is 1-dimensional and the proposition does not apply: $(x) \subseteq (x,y)$ in fact not gorenstein since it is not CM ($\dep 0$). 
\end{ex}

\begin{prop}
Let $(R,\fm,k)$ be a $d$-dimensional regular local ring and $I$ an ideal so that $R/I$ is 0-dimensional ($\sqrt{I}=\fm$). Then $R/I$ is gorenstein if and only if $\dim_k \Tor_d^R(R/I,k)=1$ (the $d$th Betti number of $R/I$). 
\end{prop}

\noindent Proof: Since $R$ is a regular local ring, $k$ is minimally resolved by the Koszul complex on a minimal set of generators $x_1,\cdots,x_d$ for $\fm$. The tail of the resolution is (the first nonzero entry occurring in the $d$th spot) up to sign
\[
0 \ma{} R \ma{} R^d \ma{} R^{\binom{d}{2}} \ma{} \cdots
\]
the first map is $(x_1,\cdots,x_d)^T$. Tensoring with $R/I$ gives the complex
\[
0 \ma{} R/I \ma{} (R/I)^d
\]
where the map is (up to sign) $(\ov{x}_1,\cdots,\ov{x}_d)$. What is the homology at $R/I$? We know that all of the $\ov{x}$'s in $R/I$ that go to 0 every component are $\{\ov{r} \in R/I \;|\; \ov{x}_i \ov{r} =0, i=1,2,\cdots,d\}=\soc R$. So $R/I$ is gorenstein if and only if $\soc R/I$ is 1-dimensional if and only if $\dim_k \Tor_d^R(R/I,k)=1$. \qed \\

\begin{ex}
If $R=k[x,y]$ and $I=(x^2,xy,y^2)$, then we have
\[
0 \ma{} R^2 \ma{} R^3 \ma{} R \ma{} R/I \ma{} 0
\]
where the first map is $\begin{pmatrix} y & 0 \\ -x & y \\ 0 & -x \end{pmatrix}$ and the second map is $\begin{pmatrix} x^2 & xy & y^2 \end{pmatrix}$. We know that the second betti number of $R/I=2$ so that $k[x,y]/(x^2,xy,y^2)$ is not gorenstein. 
\end{ex}

\begin{dfn}[(Local) Complete Intersection Ring]
A local complete intersection ring is a local ring $(R,\fm,k)$ such that
\[
\hat{R} \cong Q/(f_1,\cdots,f_c)
\]
for some (complete) regular local ring $Q$ and regular sequence $f_1,\cdots,f_c$. The codimension is $c$ (this turns out to be well-defined as long as $f_1,\cdots,f_c \in \fm^2$. 
\end{dfn}

\begin{rem}
There is a hierarchy:
\[
\text{Regular Local Rings } \supset \text{Complete Intersection Rings } \supset \text{Gorenstein Rings } \supset  \text{Cohen-Macaulay Rings}
\] We will show that these inclusions are strict. First, consider $R=k[x]/(x^2)$. This ring is 0-dimensional with maximal ideal generated by a single element so that this is a complete intersection ring but this is not a regular local ring. 

Now consider $R=k[x,y,z]/(xy,xz,yz,x^2-y^2,y^2-z^2)$. This is a 0-dimensional ring and the ideal killed needs $5>3$ generators so that this is not a complete intersection ring but this is gorenstein since $\fm^2=(x^2)$ is the socle. Finally, we have seen that $R=k[x,y]/(x^2,xy,y^2)$ is CM but is not gorenstein. 
\end{rem}

\begin{thmm}
Let $(R,\fm,k)$ be a $d$-dimensional regular local ring and $I \unlhd R$ be an ideal. Then
\begin{enumerate}[(i)]
\item if $\dep R/I=d-1$, then $I$ is a principal ideal. hence, $R/I$ is a complete intersection ring (hence $R/I$ is gorenstein. 

\item (due to Serre) if $\dep R/I=d-2$, then $R/I$ is gorenstein if and only if $R/I$ is a complete intersection ring if and only if $\mu_R(I)=2$.
\end{enumerate}
\end{thmm}

\noindent Proof:
\begin{enumerate}[(i)]
\item By Auslander-Buchsbaum, $\pd_R R/I=1$ so the free resolution is
\[
0 \ma{} R^a \ma{} R \ma{} R/I \ma{} 0
\]
for some $a \geq 1$. But a regular local ring is a domain so tensoring with the quotient field $Q$, we obtain
\[
0 \ma{} Q^a \ma{} Q \ma{} R/I \otimes_R Q
\]
But $R/I \otimes_R Q=0$ and $I \neq 0$. Therefore, $I$ contains an element which is inverted via the tensor. Hence, $a=1$ by linear algebra and the resolution is
\[
0 \ma{} R \ma{\delta} R \ma{} R/I \ma{} 0
\]
and $I$ is generated by $\delta(1)$.

\item This follows essentially the same way. We have resolution of $R/I$
\[
0 \ma{} R^a \ma{} R^b \ma{} R \ma{} R/I \ma{} 0
\]
Tensoring with $Q$ again and proceed the same way, we obtain $b=a+1$ by linear algebra over $Q$. So $R/I$ is gorenstein if and only if $a=1$ (by a previous theorem) if and only if $b=2$ if and only if $\mu_R(I)=2$ if and only if $\htt I=\mu_R(I)=2$ (since $R$ is CM) if and only if $I$ is generated by a regular sequence. 
\end{enumerate}
\qed \\

\begin{rem}
Kunz's Theorem says that an ``almost" complete intersection ring is never gorenstein, where by ``almost" we mean $Q/I$, where $Q$ is a regular local ring and $\mu_Q(I)=\htt I+1$. 
\end{rem}

We now will proceed to giving a more modern definition of gorenstein rings. We need to talk about injective modules over noetherian rings. Recall that if $R$ is a ring and $M \subseteq E$ is an $R$-module, the following are equivalent:
\begin{enumerate}[(1)]
\item $E$ is injective and there is no injective module $E'$ with $M \subseteq E' \subseteq E$
\item $E$ is injective and $M \subseteq E$ is essential
\item $E$ is a maximal essential extension of $M$.
\end{enumerate}

Furthermore, if $_R M$ is a $R$-module so that such an $E$ exists, then this $E$ is unique up to isomorphism and is called the injective hull (or injective envelope) of $M$ (see approximately Theorem 9.27 of MAT 731 notes). 

\begin{thmm}
Let $R$ be a noetherian ring.
\begin{enumerate}[(i)]
\item an injective module $E$ is indecomposable if and only if $E \cong E_R(R/p)$ for some $p \in \spec R$. 
\item every injective $R$-module is a direct sum of indecomposable injective modules.
\end{enumerate}
\end{thmm}

\noindent Proof:
\begin{enumerate}[(i)]
\item To see the reverse direction, we want to show that $E_R(R/p)$ is indecomposable for all $p \in \spec R$. Suppose $E_R(R/p) \cong M_1 \oplus M_2$. We have an injective $R/p \to E_R(R/p)$. Set $I_1=M_1 \cap R/p$ and $I_2=M_2 \cap M_2 \cap R/p$. These are nonzero since this is an essential extension so $I_1,I_2$ are nonzero ideals of the integral domain $R/p$. Hence, $0 \neq I_1I_2 \subseteq I_1 \cap I_2$, contradicting the fact that the sum is direct. 

To see the forward direction, suppose that $E$ is an indecomposable injective $R$-module. Let $p \in \ass_R E$. [Over noetherian rings, every module has associated primes, i.e. annihilators of nonzero elements that are prime. The set of annihilators is the set of ideals so it has maximal elements that one can show are prime.] Then $p=(0:_E u)$ for some $u \in E$ and so there is an injective $R/p \to E$ sending $\ov{1} \mapsto u$. This extends to $E_R(R/p) \to E$ by injectivity. This inclusion splits since $E_R(R/p)$ is injective so $E \cong E_R(R/p)$ by indecomposability. 

\item Let $E$ be an injective $R$-module. Then the proof above shows for any associated prime $p \in \ass_R(E)$, the injective hull sits inside $E$ as a direct summand of $E$. We use Zorn's lemma. Let $\Lambda$ be the set of all families $\{E_\alpha\}$ of indecomposable injective submodules of $E$ such that $\sum E_\alpha$ is a direct sum, i.e. $\sum_\alpha E_\alpha=\bigoplus_\alpha E_\alpha$ in $E$. Notice that $\Lambda \neq \emptyset$ as it contains the family $\{E_R(R/p)\}$ for any $p \in \ass_R E$. Partially order $\Lambda$ by inclusion: $\{E_\alpha\} \leq \{E_\beta\}$ if $E_\alpha \in \{E_\beta\}$ for all $\alpha$. By Zorn's Lemma, $\Lambda$ has a maximal element $\{E_\alpha\}$. If $E=\bigoplus_\alpha E_\alpha$ and we are done. If not, we know by Bass' Theorem that a direct sum of injective modules is injective so that the inclusion $\bigoplus_\alpha E_\alpha \to E$ splits, say $E=\left(\bigoplus_\alpha E_\alpha \right) \oplus N$ and $N$ is injective. By the above argument, take $q \in \ass_R N$ and get that $E_R(R/q)$ is a direct summand of $N$. Then $\{E_\alpha,E_R(R/q)\}$ contradicts the maximality of $\{E_\alpha\}$. 
\end{enumerate}
\qed \\

\begin{rem}
Note that (ii) holds for left noetherian rings even in the noncommutative case while (i) needs commutativity. We have a theorem of Bass in the 1950s: a ring $R$ is noetherian if and only if an arbitrary direct sum of injective modules is injective. 
\end{rem}

\begin{cor}
$\hat{\ass_R(E_R(R/p))} = \{p\}$. As a special case, if we have a local ring $(R,\fm,k)$, we define $E=E_R(k)$ and add this to our description: $(R,\fm,k,E)$. 
\end{cor}

\textbf{For a bit now, we will take our rings to be noetherian.}

\begin{prop}
Let $p \in \spec R$ and $x \in R$.
\begin{enumerate}[(i)]
\item if $x \notin p$, then $x$ acts as a unit on $E_R(R/p)$, i.e. multiplication by $x$ is an isomorphism. 
\item if $x \in p$, then $E_R(R/p)$ is $x$-torsion, i.e. if $u \in E_R(R/p)$, then there is a $n$ such that $x^nu=0$ (the $n$ depends on $u$ generally). 
\end{enumerate}
\end{prop}

\noindent Proof:
\begin{enumerate}[(i)]
\item Consider the map $x: E \ma{x} E$. If this is not injective, then $x$ kills something. So $(0:_E x) \neq 0$. By essentiality, $(0:_E x) \cap R/p \neq 0$ so $x$ is a zerodivisor on $R/p$, which is impossible as $R/p$ isa  domain and $x \notin p$. As $E$ is injective, the map splits. But then this must be an isomorphism since $E$ is indecomposable. 

\item Let $u \in E_R(R/p)$. So we have an injection $R_u \to E_R(R/p)$ so that $\ass_S(Ru) \subseteq \ass_R(R/p)=\{p\}$. But $R_u \cong R/\ass_R u$ is $p$-primary and $p^n \subseteq \ann_R u$ so $x^nu=0$ for some $n$. 
\end{enumerate}
\qed \\

\begin{cor}
The function $\spec R \to $ indecomposable injective modules, given by $p \mapsto E_R(R/p)$, is bijective. 
\end{cor}

\begin{prop}
For the ring $(R,\fm,k,E)$, we have $\supp E=\{\fm\}$ and $\soc E$ is 1-dimensional. 
\end{prop}

\noindent Proof: We know that $\ass_R(E) \subseteq \supp(E)$ so that $\{\fm\} \subseteq \supp E$ and $\ass_R E$ contains all minimal elements of $\supp E$. So if we had $p \subseteq \fm$ in $\supp E$. Then we must have some non maximal primes in $\ass_R E$ so that $\supp E=\{\fm\}$. 

Now $E=E_R(k)$ so that the inclusion of $k \to E$ is essential. Choose $0 \neq x \in k$. We will show $\soc E=Rx$. Certainly, $\fm x=0$ so that $x \in \soc E$. Suppose $y \in \soc E$ and $y \notin Rx$. We have $\fm y=\fm x=0$ and $Ry \cap Rx \neq 0$ as $k \to E$ is essential. If $ay=bx$ in $Ry \cap Rx$, then $a,b$ are units (since $a,b \notin \fm$) so $y=a^{-1}bx \in Rx$, a contradiction. \qed \\

\begin{dfn}[Matlis Duality Functor]
The contravariant functor on $R$-modules, $M^\vee \defeq \Hom_R(M,E)$ given by $(M \ma{f} N)^\vee \defeq N^\vee M^\vee$. (Note that we have a map $N \to E$, where $E$ is injective so that we have $M \to N \to E$, which we can lift to $M$.)
\end{dfn}

\begin{rem}
Since $E=E_R(k)$ is injective, $(-)^\vee$ is exact. Normally, $\Hom(M,-)$ is left exact only.
\end{rem}

\begin{thmm}
Suppose $(R,\fm,k,E)$ is a 0-dimensional local ring, e.g. artinian local ring, and $M$ a finitely generated $R$-module (equivalently, $M$ has finite length). Then
\begin{enumerate}[(i)]
\item $l(M^\vee)=l(M)$
\item $M^\vee=0$ if and only if $M=0$
\item $M^{\vee \vee} \cong M$
\item $l(E)=l(R)$. In particular, $E$ is finitely generated. 
\end{enumerate}
\end{thmm}

\noindent Proof:
\begin{enumerate}[(i)]
\item We proceed by induction on $l(M)$. If $l(M)=1$ then $M \cong k$. Then we have
\[
M^\vee = \Hom_R(k,E)=\Hom_R(R/\fm,E)=(0:_E \fm)=\soc E
\]
where the middle equality follows from the fact that $\Hom_T(T/I,N)=(0:_N I)$. This is 1-dimensional from the proceeding proposition. If $l(M)>1$, then we can find a submodule $M' \subset M$ with $M/M' \cong k$ so 
\[
0 \ma{} M' \ma{} M \ma{} k \ma{} 0
\]
is exact so that 
\[
0 \ma{} k^\vee \ma{} M^\vee \ma{} (M')^\vee \ma{} 0
\]
is exact. But then
\[
l(M^\vee)=l((M')^\vee)+l(k^\vee)+l(M')+l(k)=l(M)
\]
by the first short exact sequence. 

\item $l(M^\vee)=0$ if and only if $l(M)=0$ is equivalent by (i).

\item We have a map $\theta: M \to M^{\vee \vee}=\Hom_R(\Hom_R(M,E),E)$ sending $u \in M$ to $\text{ev}_u: \Hom_R(M,E) \to E$ (the evaluation map): $\text{ev}_u(f)=f(u)$. [One should check that this is a homomorphism.] We know from (i) that $l(M^{\vee \vee})=l(M^\vee)=l(M)$. To show $\theta$ is an isomorphism it suffices to show that it is injective or surjective. Suppose $\theta(u)=0$. Then $\theta(u)=\text{ev}_u=0$ so that $f(u)=0$ for all $f \in M^\vee$. We know that $Ru \leq M$ gives a short exact sequence
\[
0 \ma{} Ru \ma{} M \ma{} M/Ru \ma{} 0
\]
so that we have
\[
0 \ma{} (M/Ru)^\vee \ma{} M^\vee \ma{\beta} (Ru)^\vee \ma{} 0
\]
The map $\beta$ is just the restriction to the submodule $Ru \leq M$. So $\beta(f)=f|_{Ru}=0$ for all $f \in M^\vee$. But then $\beta=0$ so that $(Ru)^\vee=0$ so that $Ru=0$ by (ii). But then $u=0$ showing that this is injective. \qed \\

\item $E=\Hom_R(R,E)=R^\vee$ so that $l(E)=l(R)$.

\end{enumerate}
\qed \\

\begin{rem}
We have generally that $M^\vee \not\cong M$. As an example, take $R=k[x,y]/(x^2,xy,y^2)$ has a 2-dimensional socle, $(x,y)$, while $R^\vee=E$ has a 1-dimensional socle by the proposition so that $R \not\cong E$. 
\end{rem}

\begin{rem}
There is another obvious ``duality" $(-)^*=\Hom_R(-,R)$. This has two problems: First, it is not exact as $R$ need not be injective. Second, this need not be an involution: $(-)^{**}=(-)$. Why? Suppose $R$ is a 0-dimensional local ring with socle having dimension $t$ as a vector space: $\dim_k \soc R=t$. Then 
\[
(k)^*=\Hom_R(R/\fm,R)=\soc R=k^t
\]
so $k^(**)=\Hom_R(k^t,R)=\Hom_R(k,R)^t=k^{t^2}$ can only be isomorphic to $k$ if $t=1$, so $R$ is gorenstein. 
\end{rem}

\begin{thmm}
The following are equivalent for a 0-dimensional ring $(R,\fm,k,E)$:
\begin{enumerate}[(i)]
\item $R$ is gorenstein
\item $R \cong E$
\item $R$ is self-injective, i.e. $R$ is injective as an $R$-module
\end{enumerate}
\end{thmm}

\noindent Proof:
\begin{enumerate}
\item[(i)$\to$(ii):] We know that $\soc R \leq R$ is essential and as $R$ is gorenstein, $\soc R$ is 1-dimensional - $\soc R=k$. So we have $k \subseteq R \subseteq E_R(k)$. But $l(R)=l(E)$ so that $R=E$.

\item[(ii)$\to$(iii):] $E$ is injective so that we are done.

\item[(iii)$\to$(i):] Injective modules over $R$ are direct sums of $E_R(R/p)$. But $\spec R=\{\fm\}$. So if $R$ is injective, then $R \cong E^t$. But $l(R)=l(E^t)$ and $l(R)=l(E)$ so that $t=1$. 
\end{enumerate}
\qed \\

This gives us the modern definition of gorenstein:

\begin{dfn}[Gorenstein]
A noetherian local ring $R$ is called gorenstein if $R$ is self-injective, i.e. $R$ is injective as an $R$-module. 
\end{dfn}

Note that the previous theorem results are \emph{not} equivalent for noncommutative artinian local rings.

\begin{cor}
If we have a noetherian local ring, $(R,\fm,k,E)$, then $R$ is gorenstein if and only if $\id_R R$ is finite. 
\end{cor}

\noindent Proof: For the forward direction, note that $0<\infty$. For the reverse direction, suppose that $\id_R R=s$. We resolve $R$:
\[
0 \ma{} R \ma{} I^0 \ma{} I^1 \ma{} \cdots \ma{} I^s \ma{} 0
\]
We know that $I^j$ is a direct sum of $E_R(R/p)$. But $\spec R=\{\fm\}$ so the $I$'s are a direct sum of $E$'s. Then we have
\[
0 \ma{} R \ma{} E^{\mu_0} \ma{} E^{\mu_1} \ma{} \cdots \ma{} E^{\mu_s} \ma{} 0
\]
Apply $\Hom_R(-,E)$ and obtain the exact sequence
\[
0 \ma{} \left(E^\vee\right)^{\mu_s} \ma{} \left(E^\vee\right)^{\mu_{s-1}} \ma{} \cdots \ma{} \left(E^\vee\right)^{\mu_0} \ma{} R^\vee \ma{} 0
\]
But this is exactly
\[
0 \ma{} R^{\mu_s} \ma{} R^{\mu_{s-1}} \ma{} \cdots \ma{} R^{\mu_0} \ma{} E \ma{} 0
\]
so that $E$ has finite projective dimension. Thus $\pd_R E<\infty$. By Auslander-Buchsbaum, $\pd_R E+\dep E=\dep R$. But $\dep E=\dep R=0$ so that $E$ is a projective $R$-module. Hence, $E \cong R^t$ (as projectives over a local ring are free). But since $\lambda(E)=\lambda(R)$, we have $t=1$ so that $R$ is gorenstein. \qed \\

But what if $\dim R>0$? 

\begin{ex}
Let $R=k[t]_{(t)}$. What is $E_R(R/\fm)$? We claim that $E_R(R/\fm) \cong k[t^{-1}]$ which is very much not finitely generated since this action moves things up in degree. So looking at the lowest power of a possible generating set only generates that power or higher. Hence, this cannot be finitely generated. Note that the $R$-module structure is given by $t \cdot t^{-a} \defeq t^{-a+1}$ unless $a=0$ in which case we define this to be 0. 

To prove this claim, note that $\soc(k[t^{-1}])=(0:_{k[t^{-1}]} t)=k$ is 1-dimensional. To see that this is injective, we use Baer's criterion. If $\phi: I \to k[t^{-1}]$ is a map, where $I \unlhd R$, extend $\phi$ to $\tilde{\phi}: R \to k[t^{-1}]$. But we know all the ideals of $R$ (as we have turned all polynomials with nonzero constant term into a unit). Every ideal of $R$ is of the form $(t^i)$. So $\phi: (t^i) \to k[t^{-1}]$ via $t^i \mapsto \sum_{j=0}^n a_j t^{-j}$ and $\tilde{\phi}: R \to k[t^{-1}]$ by $\tilde{\phi}(1)=\sum_{j=0}^n a_j t^{-j-i}$. Then one needs to check $\tilde{\phi}(t^i)$ is as above
\[
\tilde{\phi}(t^i)=t^i\tilde{\phi}= t^i \sum_{j=0}^n a_j t^{-j-i}= \sum_{j=0}^n a_j t^{-j} 
\]
\end{ex}

\begin{lem}
$E_{R/I}(k)=\Hom_R(R/I,E_R(k))$
\end{lem}

\noindent Proof: We need to check that $\Hom_R(R/I,E_R(k))$ contains $k$ as its socle and is injective over $R/I$. 
\[
\begin{split}
\soc(\Hom_R(R/I,E))&=\Hom_R(k,\Hom_R(R/I,E)) \\
&=\Hom_R(k \otimes_R R/I,E) \\
&=\Hom_R(k,E) \\
&=\soc E \\
&=k
\end{split}
\]
where the second equality follows from the Adjoint Isomorphism Theorem. For injectivity, we need to show that $\Hom_{R/I}(-,\Hom_R(R/I,E_R(k)))$ is an exact functor on $R/I$-modules. But this is precisely $\Hom_R(-\otimes_{R/I} R/I,E)=\Hom_R(-,E)$. \qed \\

\begin{thmm}
Let $(R,\fm,k,E)$ be a noetherian local ring. Then $R^{\vee \vee}\defeq \Hom_R(E,E)=\hat{R}$.
\end{thmm}

\noindent Proof: We already know this in the 0-dimensional case: $\hat{R}= \plim R/\fm^n$. So we need show that the $\Hom$ is also the inverse limit. Set $E_n=(0:_E \fm^n) \subseteq E$. Then we have $E_n \cong \Hom_R(R/\fm^n,E)$. By the lemma, this is $E_{R/\fm^n}(k)$. For $f \in \Hom_R(E,E)$, set $f_n=f|_{E_n}$. Notice that if $u \in E_n$, then $\fm^nu=0$ so $\fm^nf(u)=f(\fm^nu)=0$. So $f(E_n) \subseteq E_n$. Therefore, we can consider $f_n=f|_{E_n}$ as being in $\Hom_R(E_n,E_n)$. 

Observe that $\Hom_R(E_n,E_n)=\Hom_{R/\fm^n}(E_n,E_n)$ so the target of $\phi$ is really $\plim \Hom_{R/\fm^n}(E_n,E_n)=\plim \Hom_{R/\fm^n}(E_{R/\fm^n}(k),E_{R/\fm^n}(k))=\plim R/\fm^n=\hat{R}$ by Matlis duality in the 0-dimensional case (note we have also used the fact that $R/\fm^n$ is an artinian ring). Now define $\phi: \Hom_R(E,E) \to \plim \Hom_R(E_n,E_n)$ via $f \mapsto (f_0,f_1,\cdots)$. It only remains to show that $\phi$ is an isomorphism. First, we show that $\phi$ is well defined. Take $f_n|_{E_m}=f|_{E_n}|_{E_m}$ for $m \leq n$. Now $\phi$ is a homomorphism because it is a restriction of a homomorphism. We see that $\phi$ is injective as $\ass_R E=\{\fm\}$ and that every $u \in E$ is killed by some power of $\fm$ so that $E$ is $\cup E_n$ so that if $f_n=0$ for large $n$ then $f=0$. Finally, if $(g_0,g_1,\cdots)$ is a sequence of amps $g_n: E_n \to E_n$ with $g_n|_{E_m}=g_n$. Define $g: E \to E$ using $E=\cup_n E_n$. This shows that $\phi$ is surjective. \qed \\

\begin{lem}
For any left $R$-module $M$, $M^\vee=0$ if and only if $M=0$. 
\end{lem}

\noindent Proof: We have already shown this for finitely generated $M$ over artinian local rings. If $M$ is a module with $M^\vee=0$, take $x \in M$ so that the inclusion $Rx \to M$ and the projection $M^\vee \to (Rx)^\vee$. But since $M^\vee=0$, we get $(Rx)^\vee=0$ so that if we prove that this implies $Rx=0$ then $x=0$ for all $x \in M$ and we are done. This allows us to reduce down to the case where $M$ is finitely generated. If $M$ is finitely generated, then $M/\fm M\neq 0$ by Nakayama's Lemma. Then we have surjections $M \to M/\fm M \to k$. But then $k^\vee=k$ injects into $M^\vee$ so that $M^\vee \neq 0$. \qed \\

\begin{lem}
If $(R,\fm,k,E)$ is a noetherian local ring, then $E$ is artinian. 
\end{lem}

\noindent Proof: Let 
\[
\cdots \subseteq D_{n+1} \subseteq D_n \subseteq E
\]
be a descending chain of submodules of $E$. We know that $E^\vee=\hat{R}$ so that we have surjections
\[
R \ma{} \cdots \ma{} D_n^\vee \ma{} D_{n-1}^\vee \ma{} \cdots
\]
It follow that $D_n^\vee \cong \hat{R}/I_n$ for some ideal $I_n=\ker(\hat{R} \to D_n^\vee)$. We have $I_n \subseteq I_{n+1}$ for all $n$. But $\hat{R}$ is noetherian so that $I_n=I_{n+1}$ for some $n$. Now we have the short exact sequence
\[
0 \ma{} D_{n+1} \ma{} D_n \ma{} D_n/D_{n+1} \ma{} 0
\]
Then
\[
0 \ma{} (D_n/D_{n+1})^\vee \ma{} D_n^\vee \ma{} D_{n+1}^\vee \ma{} 0
\]
Now since we have an isomorphism $D_n^\vee \to D_{n+1}^\vee$ for some $n$, we must have $(D_n/D_{n+1})^\vee=0$ so $D_n/D_{n+1}=0$ by the preceding lemma so that $D_n \cong D_{n+1}$. \qed \\

\begin{rem}
Any artinian $R$-module is naturally an $\hat{R}$-module: if $N$ is artinian then $Rx \subseteq N$ has finite length for any $x \in N$. Hence, $\fm^n x=0$ for some $n$. Then if you have $\hat{r}=r_0+r_1+r_2+\cdots \in \hat{R}$ with $r_i \in \fm^i$ for all $i$. Set $\hat{r} x=(r_0+r_1+\cdots+r_{n-1})x$. One needs only verify that this is well defined. 
\end{rem}

\begin{thmm}[Matlis Duality]
Let $(R,\fm,k,E)$ be a noetherian local ring. Then there is a bijection between artinian $R$-modules and noetherian $\hat{R}$-modules, given by $(-)^\vee$ in both directions. Note that $(-)^{\vee\vee}$ is the identity map for either direction. 
\end{thmm}

\noindent Proof: Let $N$ be an artinian $R$-module. Then $N \subseteq E^r$ for some $r$. Then we have a surjection $\hat{R}^r \to N^r$ so that $N$ is a finitely generated (hence noetherian) $\hat{R}$-module. Conversely, if $M$ is a finitely generated $\hat{R}$-module, then we have a surjection $\hat{R}^r \to M$ (which we must check). First, note that we have an injection $M^\vee \to \Hom_R(\hat{R},E)=\Hom_R(R,E)^r=E^r$. We know that $E$ is a $\hat{R}$-module by the proceeding remark. Then $E=\Hom_{\hat{R}}(\hat{R},E) \subseteq \Hom_R(\hat{R},E)$. But this is actually an equality! For any $f \in \Hom_R(\hat{R},E)$, $f(1)$ is killed by $\fm^n$ so that $\fm^n f=0$ so that $f$ is $\hat{R}$-linear. But then $\Hom_R(\hat{R},E)=E$. 

We know that the alleged bijection $E \to \hat{R}$ is at least well defined and is given by $(-)^\vee$. It is enough to show that both compositions give the identity. Recall that $\theta_M: M \to M^{\vee\vee}$ is given by $x \mapsto \text{ev}_x$. If $M$ is finitely generated over $\hat{R}$, it has a presentation by free modules, say
\[
\hat{R}^B \ma{} \hat{R}^A \ma{} M \ma{} 0
\]
Check this sequence twice and we obtain
\[
\begin{tikzcd}
\hat{R}^B \arrow{d}{\theta_{\hat{R}^B}} \arrow{r} & \hat{R}^A \arrow{d}{\theta_{\hat{R}^A}} \arrow{r} & M \arrow{d}{\theta_M} \arrow{r} & 0 \\
(\hat{R}^B)^{\vee\vee} \arrow{r} & (\hat{R}^A)^{\vee\vee} \arrow{r} & M^{\vee\vee} \arrow{r} & 0
\end{tikzcd}
\]
We know that $\theta_{\hat{R}^B} \to \theta_{\hat{R}^A}$ are isomorphisms so that $\theta_M$ is an isomorphism. Similarly, if $N$ is a left $R$-module that is artinian, we obtain
\[
0 \ma{} N \ma{} E^r \ma{} E^s
\]
(the cokernel is artinian so it embeds in $E^s$ for some $s$). One then obtains a similar diagram for $E^S$. Since $\theta_E$ is an isomorphism, so too is $\theta_N$. \qed \\

\begin{cor}
A $R$-module $N$ is artinian if and only if $N \subseteq E^r$ for some $r$. 
\end{cor}

\noindent Proof: For the forward direction, recall that any artinian module is essential over its socle. If $N$ is artinian, $\soc N$ is artinian. Therefore, $\soc N$ is a vector space with the descending chain condition, which means it is finite dimensional. Say that $\soc N \cong k^r$. Then $k^r=\soc N$ is an essential submodule of $N$ so that $N \subset E_R(k^r)=E_R(k)^r$, as desired. The reverse direction was shown in the preceding lemma as submodules of artinian modules are artinian. \qed \\

\subsection{Higher Dimensional Gorenstein Rings}

Our goal is to show that $(R,\fm)$ is a gorenstein local ring if and only if $\id_R$ is finite. We already know this in the case where $\dim R=0$. We begin with a generally useful lemma.

\begin{lem}
Let $R$ be a noetherian ring and $M$ a finitely generated $R$-module. Then $M$ has a finite filtration 
\[
0=M_0 \subset M_1 \subset \cdots \subset M_n=M
\]
such that $M_i/M_{i-1} \cong R/p_i$ for some $p_1,\cdots,p_n \in \supp M$. 
\end{lem}

\noindent Proof: Since $\ass_R M \neq 0$, we can set $M_1=R/p$ for any $p \in \ass_R M$. At the $i$th stage, set $M_i=R/p_i$ for some $p_i \in \ass_R(M/M_{i-1})$. This creates an ascending chain which terminates by the ascending chain condition so that we must have $M_n=M$ for some $n$. Hence we have an injection $R/p_i \to S$, where $S$ is a quotient of $M$. Then when we localize we get an injection $k_{(p_i)} \to S_{(p_i)}'$, where $S'$ is a quotient of $M_{(p_i)}$. This cannot happen if $M_{(p_i)} \neq 0$. \qed \\

\begin{rem}
$p_1 \in \ass_R M$ but for $i>1$, there is no reason to have $p_i \in \ass_R M$. 
\end{rem}

\begin{prop}
If $R$ is noetherian and $M,N$ are finitely generated $R$-modules, if $\Ext_R^1(R/p,M)=0$ for al $p \in \supp N$, then $\Ext_R^i(N,M)=0$.
\end{prop}

\noindent Proof: Filter $N$ with successive quotients $R/p$ for some $p \in \supp N$ and use the long exact sequence for Ext. \qed \\

\begin{cor}
$\id_R M \leq n$ if and only if $\Ext_R^{n+1}(R/p,M)=0$ for all $p \in \spec R$. 
\end{cor}

\begin{lem}
Let $p \in \spec R$. If $\Ext_R^n(R/p,M) \neq 0$, then $\Ext_R^{n+1}(R/q,M) \neq 0$ for all $q \supsetneq p$.
\end{lem}

\noindent Proof: We proceed by contrapositive. If $\Ext_R^{n+1}(R/p,M)=0$ with $q \supsetneq p$, then $\Ext_R^n(R/p,M)=0$. Take $x \in q \setminus p$. We get a short exact sequence
\[
0 \ma{} R/p \ma{x} R/p \ma{} R/(p,x) \ma{} 0
\]
This gives a long exact sequence in $\Ext(-,M)$
\[
\Ext_R^n(R/(p,x),M) \ma{} \Ext_R^n(R/p,M) \ma{x} \Ext_R^n(R/p,M) \ma{} \Ext_R^{n+1}(R/(p,x),M)
\]
Now $q \supsetneq (p,x)$ so $q \in \supp(R/(p,x))$. But then $\Ext_R^{n+1}(R/(p,x),M)=0$. Therefore, $\Ext_R^n(R/p,M) \ma{x} \Ext_R^n(R/p,M)$ is surjective so $\Ext_R^n(R/p,M)=0$ by Nakayama's Lemma. \qed \\

\begin{prop}
If $(R,\fm)$ is a local noetherian ring and $M$ a finitely generated $R$-module, then $\id_R M=\sup\{i\;|\; \Ext_R^i(R/\fm,M) \neq 0\}$.
\end{prop}

\noindent Proof: We know by a previous result that $\id_R M \geq \sup\{i\;|\; \Ext_R^i(R/\fm,M) \neq 0\}$. It only remains to prove the other direction. If $n \geq \sup\{i\;|\; \Ext_R^i(R/\fm,M) \neq 0\}$, then by the Lemma, $n \geq \sup\{i\;|\; \Ext_R^i(R/p,M) \neq 0\}$ for some $p$. \qed \\

Really that $\dep M=\inf\{i\;|\; \Ext_R^i(R/\fm,M) \neq 0\}$ so $\dep M \leq \id_R M$. In fact, if $\id_R M <\infty$, then $\dim M \leq \id_R M=\dep R$. 

\begin{lem}
Let $R$ be a ring and $x \in R$. Let $M,N$ be left $R$-modules. Assume that $x$ is a nonzerodivisor on $M$ and $R$ and that $xN=0$, then 
\begin{enumerate}[(i)]
\item $\Hom_R(N,M)=0$
\item For all $i \geq 0$, $\Ext_R^{i+1}(N,M) \cong \Ext^i_{R/x}(N,M/xM)$. 
\end{enumerate}
\end{lem}

\noindent Proof:
\begin{enumerate}[(i)]
\item If $\phi: N \to M$ is a nonzero map, then $\phi(n) \neq 0$ for some $n \in N$. But $x\phi(n)=\phi(xn)=\phi(0)=0$, a contradiction. Therefore, $x$ is a nonzerodivisor on $M$.
\item Define a functor from $R/(x)$-modules to $R/(x)$-modules by $T^i(-)=\Ext_R^{i+1}(-,M)$. To show the isomorphism, we will show that the $T$'s are right derived functors of $\Hom_{R/(x)}(-,M/xM)$. 

Note that $T^0(N)=\Ext_R^1(N,M)$. We have a short exact sequence
\[
0 \ma{} M \ma{x} M \ma{} M/xM \ma{} 0
\]
Then we get an associated long exact sequence
\[
0 \ma{} \Hom_R(N,M) \ma{x} \Hom_R(N,M) \ma{} \Hom_R(N,M/xM) \ma{x} \Ext_R^1(N,M) \ma{} \Ext_R^1(N,M)
\]
Now $x$ kills $N$ so that it kills $\Ext_R^1(N,M)$, so the multiplication by $x$ map above is the zero map. We know that $\Hom_R(N,M)=0$ so that we get an isomorphism $\Hom_R(N,M/xM)=\Hom_{R/(x)}(N,M/xM) \cong \Ext_R^1(N,M) =T^0$.

Now we need show that the $T^i(-)$ vanish on projectives for $i>0$. We have a short exact sequence
\[
0 \ma{} R \ma{x} R \ma{} R/(x) \ma{} 0 
\]
But $\pd_R R/(x) \leq 1$ so that $T^i(R/(x))=\Ext_R^{i+1}(R/(x),M)=0$ for $i \geq 1$ so that $T^i$ vanishes on $R/(x)$-projectives. 

Finally, we need to show that $T^i(-)$ gives a long exact sequence from short exact sequences. But this is immediate since it is given by $\Ext$. The $T^i(-)$ form a strongly connected sequence of functors, which vanish on projectives, and agree with $\Hom_{R/(x)}(-,M/xM)$ for $i>0$ so it must be $\Ext^i_{R/(x)}(N,M/xM)$. 
\end{enumerate}
\qed \\

\begin{cor}
If $(R,\fm)$ is a local ring, $M$ is a finitely generated $R$-module, and $x \in \fm$ is a nonzerodivisor on $R$ and $M$, then $\id_{R/(x)}M/xM=\id_R M-1$.
\end{cor}

\noindent Proof: We know that $\id_R M=\sup\{i \;|\; \Ext^i_R(R/\fm,M) \neq 0\}$. Therefore by the lemma using Auslander-Buchsbaum, we have $\sup\{i\;|\; \Ext_R^{i-1}(\ov{R}/\ov{\fm},\ov{M} \neq 0\}=\id_{\ov{R}} \ov{M}+1$. (Note the $+1$ occurs because of the $i$ shift.) \qed \\

\begin{cor}
If $(R,\fm,k)$ is a gorenstein local ring then $\id_R R=\dim R$.
\end{cor}

\noindent Proof: We know that $R$ is CM so take a regular sequence $x_1,\cdots,x_d \in \fm$, where $d=\dim R$. Then $R/(x_1,\cdots,x_d)$ is a 0-dimensional gorenstein ring. But then this is self-injective so that $\id_{R/(x_1,\cdots,x_d)} R/(x_1,\cdots,x_d)=0$. But the dimension drops by 1 for each regular element killed. But then $\id_R R=\dim R$. \qed \\

\begin{thmm}
Let $(R,\fm,k)$ be a noetherian local ring and $M$ a finitely generated $R$-module. If $\id_R M< \infty$ then $\dim M \leq \id_R M=\dep R$. 
\end{thmm}

\begin{cor}
(Taking $M=R$) If $\id_R R<\infty$, then $R$ is gorenstein. 
\end{cor}

\noindent Proof: By the theorem, $\dim R \leq \id_R R=\dep R \leq \dim R$ so that $R$ is CM. Cutting down by a maximal regular sequence, $\id_{R/(\vec{x})} R/(\vec{x})=0$ so by the lemma so that $R/(\vec{x})$ is gorenstein. But then $R$ is gorenstein. \qed \\