% !TEX root = ../../commutative_algebra.tex

\newpage
\section{Integral Extensions}

\begin{dfn}[Integral]
Let $R \subseteq S$ be commutative rings with identity. An element $s \in S$ is integral over $R$ if $s$ satisfies a monic polynomial in $S$ over $R$, i.e.
\[
s^n+r_1s^{n-1}+\cdots+r_{n-1}s+r_n=0
\]
for some $n \geq 0$ and $r_i \in R$. 
\end{dfn}

\begin{ex}
If $R,S$ are field, then $s \in S$ is integral over $R$ if and only if $s$ is algebraic over $R$.
\end{ex}

\begin{ex}
Take $R=k[t^2] \subseteq k[t]=S$. Then $t \in S$ is integral over $R$ since $t^2-t=0$. More clearly, $t$ is a root of $x^2-t^2 \in R[x]$. 
\end{ex}

\begin{ex}
Take $R=k[s^3,t^4] \subseteq k[s^3,t^4,st]=S$. Then $st$ is integral over $R$ as it is the root of $x^{12}-(s^3)^4(t^4)^3$. 
\end{ex}

\begin{rem}
Recall that if $R,S$ are both fields and $s \in S$ is algebraic over $R$ then $R(s)=R[s]$. We will prove something similar for integral elements. 
\end{rem}

\begin{prop}
Let $R \subset S$ and $u \in S$. Then the following are equivalent:
\begin{enumerate}[(i)]
\item $u$ is integral over $R$
\item $R[u]$ is a finitely generated $R$-module
\item there is a subring $B$ of $S$ with $R[u]\subseteq B \subseteq S$ which is a finitely generated $R$-module.
\item there is a faithful $R[u]$-module which is finitely generated over $R$
\end{enumerate}
\end{prop}

\begin{dfn}[Integral Extension]
A ring extension $R \subseteq S$ is an integral extension if every $s \in S$ is integral over $R$.
\end{dfn}

\begin{cor}
The following are equivalent for a ring extension $R \subset S$ and $u \in S$:
\begin{enumerate}[(i)]
\item $u$ is integral over $R$
\item $R[u]$ is a finitely generated $R$-module
\item $R \subseteq R[u]$ is an integral extension. 
\end{enumerate}
\end{cor}

\begin{rem}
If $R \subseteq S \subseteq T$ and $S$ is a finitely generated $R$ module and $T$ is a finitely generated $S$-module, then $T$ is a finitely generated $R$-module. This is simple enough to show: if $S$ is generated by $x_1,\cdots,x_n$ and $T$ is generated by $y_1,\cdots,y_m$, then $T$ is $\sum Rx_i y_j$. 
\end{rem}

\begin{cor}
Let $R \subseteq S$ be a ring extension and $u_1,\cdots,u_n \in S$. Then the following are equivalent:
\begin{enumerate}[(i)]
\item $u_1,\cdots,u_n$ are integral over $R$
\item $R[u_1,\cdots,u_n]$ is a finitely generated $R$-module
\item $R \subseteq R[u_1,\cdots,u_n]$ is integral
\end{enumerate}
\end{cor}

\begin{cor}
Let $S$ be a finitely generated $R$-algebra. Then $S$ is integral over $R$ if and only if $S$ is a finitely generated $R$-module. 
\end{cor}

\begin{ex}
Let $R=k[s^6,s^4t^2,s^2t^{112},t^{191}] \subseteq k[s,t]=S$. Now $S$ is generated as an $R$-algebra by $s,t$ which are both integral (look at $x^7-s^7, x^{191}-t^{191}$) so $R \subseteq S$ is integral by the corollary and hence the module is finitely generated. 
\end{ex}

\begin{dfn}
Let $R \subseteq S$ be rings. We say that $R$ is integrally closed in $S$ if no element of $S \setminus R$ is integral over $R$. 
\end{dfn}

\begin{prop}
Let $R \subseteq S$ be rings. Set $T=\{s \in S\;|\; s \text{ integral over }R\}$. Then 
\begin{enumerate}[(i)]
\item $T$ is a subring of $S$.
\item $T$ is integrally closed in $S$.
\end{enumerate}
$T$ is called the integral closure of $R$ in $S$.
\end{prop}

\noindent Proof: 
\begin{enumerate}[(i)]
\item It is enough to show that $T$ is closed under multiplication and addition. Let $u,v \in T$. Then $R[u,v] \subseteq T$ and $R[u,v]$ is a subring which is integral over $R$ so $u+v,uv \in T$. 

\item Take $u \in S$ to be integral over $T$. Then $u^n+t_1u^{n-1}+\cdots+t_n=0$ for some $t_i \in T$. Set $A=R[t_1,\cdots,t_n]$. Then $A$ is integral over $R$ and is finitely generated as an $R$-algebra. So $A$ is a finitely generated $R$-module. Similarly, $A[u]$ is integral over $A$ and a finitely generated $A$-algebra so it is a finitely generated $A$-module. But by a previous remark, $A[u]$ is a finitely generated $R$-module. But then $u$ is integral over $R$ and so $u \in T$.
\end{enumerate}
\qed \\

\begin{ex}[Classical Example]
Let $F$ be a number field (a finite algebraic extension of $\Q$). Let $R$ be the integral closure of $\Z$ in $F$ (an ``order"). We call this $R$ a ring of algebraic integers. 
\end{ex}

\begin{ex}
Let $F=\Q(i)$, then $R=\Z[i]$.
\end{ex}

\begin{ex}
Let $d \in \Z$ be squarefree and $F=\Q(\sqrt{d})$. Then 
\[
R=\Z[\sqrt{d}]= 
\begin{cases}
\Q(n), & d \equiv 2,3 \mod 4 \\
\Z\left[\frac{1+\sqrt{d}}{2}\right], & d \equiv 1 \mod 4
\end{cases}
\]
\end{ex}

\begin{dfn}
Let $R$ be a domain with fraction field $F$. We say that $R$ is integrally closed (or normal) if $R$ is integrally closed in $F$. 
\end{dfn}

\begin{prop}
UFDs are integrally closed (or normal)
\end{prop}

\noindent Proof: Let $R$ be a UFD with field of fraction $F$. Let $x/y \in F$ be integral over $R$ and be in reduced form. We have
\[
\left(\frac{x}{y}\right)^n+r_1\left(\frac{x}{y}\right)^{n-1}+\cdots+r_{n-1}\left(\frac{x}{y}\right)+r_n=0
\]
for $r_i \in R$. Clearing denominators yields
\[
x^n+(r_1y)x^{n-1}+\cdots+r_{n-1}y^{n-1}x+r_ny^n=0
\]
So $x^n$ is divisible by $y$ so that $y$ is a unit. But then $x/y \in R$. \qed \\

\begin{ex}
Both $\Z,k[x_1,\cdots,x_n]$ are normal. 
\end{ex}

\begin{ex}
Let $R$ be the integral closure of $\Z$ in $\Q(i)$. We clearly have $\Z[i] \subseteq R$. We also know that $\Z[i]$ is a Euclidean Domain, hence a UFD. But then $\Z[i]$ is integrally closed in $\Q(i)$. So any element $r \in R$ is integral over $\Z$ and hence integral over $\Z[i]$. But then it is integral in $\Z[i]$. Therefore, $R=\Z[i]$. 
\end{ex}

\begin{ex}
Let $R=k[s^2,st,t^3]$. Is $R$ normal or not? We need to identify the fraction field of $R \subseteq k[s,t]$. We know that $Q(R)\subseteq k(s,t)$ so that 
\[
\frac{1}{t}= \left( \frac{(st)^4}{(s^2)^2t^3} \right)^{-1} \in Q(R)
\]
and 
\[
\frac{1}{s}=\left( st \cdot \frac{1}{t} \right)^{-1} \in Q(R)
\]
so $Q(R)=k(s,t)$. But $s \in Q(R) \setminus R$ is a root of $x^2-s^2$ so that $R$ is not a normal. In fact, the integral closure of $R$ [in $Q(R)$] is the whole ring $k[s,t]$. 
\end{ex}

\subsection{Primes in Integral Extensions}

It is our goal now to show that $\dim R=\dim S$ if $R \subset S$ is integral. 

\begin{lem}
Let $R \subset S$ be an integral extension of domains. Then $R$ is a field if and only if $S$ is a field (if and only if $\dim =0$). 
\end{lem}

\noindent Proof: For the forward direction, suppose that $R$ is a field and $u \in S \setminus \{0\}$. We want to show that $u$ is a unit. We have $u^n+r_1u^{n-1}+\cdots+r_n=0$ for some $r_i \in R$. If $r_n=0$, we have $u(u^{n-1}+r_1u^{n-2}+\cdots r_{n-1}u)=0$. We can cancel $u$ and proceed inductively until we have shown $u$ is a unit. Therefore, we can assume without loss of generality that $r_n \neq 0$. But $r_n$ is a unit of $R$ and so we can write $u^n+r_1u^{n-1}+\cdots+r_{n-1}u= -r_n$. But the left is $u(u^{n-1}+r_1u^{n-2}+\cdots r_{n-1}u)$ so that $u$ is a unit. 

For the reverse direction, suppose that $S$ is a field. Let $r \in R \setminus \{0\}$. Then $0 \neq r \in R \subset S$ so that $r$ is a unit in $S$. So $1/r \in S$. But then $1/r$ is integral over $R$ so that 
\[
\left(\frac{1}{r} \right)^n+a_1\left(\frac{1}{r} \right)^{n-1}+\cdots+a_{n-1}\left(\frac{1}{r} \right)+a_n=0
\]
Clearing denominators, we have
\[
1+a_1 r+a_2r^2+\cdots+a_{n-1}r^{n-1}+a_nr^n=0
\]
But then $-1=r(a_1+a_2r+\cdots+a_n r^{n-1})$ so that $r$ is a unit. \qed \\

\begin{lem}
Let $R \subseteq S$ be an integral extension and $I \subseteq S$ an ideal. If $U \subseteq R$ is multiplicatively closed then
\begin{enumerate}[(i)]
\item $R/I \cap R \subseteq S/I$ is integral 
\item $R_U \subseteq S_U$ is integral 
\end{enumerate}
\end{lem}

\noindent Proof: We prove (i). Let $\ov{u}=u+I \in S/I$. Then $u$ is integral over $R$ so that $u^n+r_1u^{n-1}+\cdots+r_n=0$. Passing to $S/I$, we get 
\[
\ov{u}^n+\ov{r}_1 \ov{u}^{n-1}+\cdots+\ov{r}_n=0
\]
so that $\ov{u}$ is integral over $R/I \cap R$. \qed \\

\begin{cor}
Let $R \subseteq S$ be an integral extension and $q \in \spec S$. Then $q$ is maximal if and only if $q \cap R$ is maximal. 
\end{cor}

\noindent Proof: By the lemma, we know that $R/q \cap R \subseteq S/q$ is integral and they are both domains. So by the other lemma, we know that $q \cap R$ is maximal if and only if $R/q \cap R$ is a field which is equivalent to $S/q$ being a field which is equivalent to $q$ being maximal. 

\subsection{Lying Over}

\begin{thmm}[Cohen-Seidenberg Lying Over]
Let $R \subseteq S$ be an integral extension and $p \in \spec R$. Then there is a $q \in \spec S$ lying over $p$, i.e. $q \cap R=p$. 
\end{thmm}

\noindent Proof: First, we localize. Let $U=R \setminus p$. Then $R_U \subseteq S_U$ is integral by the lemma and $R_U=R_p$ is local with maximal ideal $p$. Choose any maximal ideal $q \in S_U$. We know that $\spec S_U=\{q \in \spec S \;|\; q \cap U=\emptyset\}$. We claim that $q_U \cap R_U=(q \cap R)_U$ for any $q \in \spec S$. If we can only prove this, we will be done as we know that if $q_U$ is maximal then $q_U \cap R_U$ is maximal in $R_U$ so that $q_U \cap R_U=p$ so that $(q \cap R)_U=p$ showing $q \cap R=p$. We need show the claim. We have an exact sequence
\[
0 \ma{} R/q \cap R \ma{} S/q
\]
Localizing gives
\[
0 \ma{} (R/q \cap R)_U \ma{} (S/q)_U
\]
which is
\[
0 \ma{} R_U/(q \cap R)_U \ma{} S_U/q_U
\]
But the kernel is precisely $q_U \cap R_U$ so that $q_U \cap R_U=(q\cap R)_U$. \qed \\

\begin{lem}[Incomparable]
Let $R \subseteq S$ be an integral extension and $q \subsetneq Q$ be primes of $S$. Then $q \cap R \subsetneq Q \cap R$.
\end{lem}

\noindent Proof: Clearly $q \cap R \subseteq Q \cap R$. Suppose that $q \cap R = Q \cap R=p \in \spec R$. Set $U=R \setminus p$ and localize at $U$. We know that $P_U= pR_p$ is the unique maximal ideal of $R_U$ and $S_U$ is integral over $R_U$. But $q_U \cap R_U=(q \cap R)_U=p_U$ and $Q_U \cap R_U=(Q \cap R)_U=Q_U$ so that $q_U$ and $Q_U$ are maximal ideals of $S_U$. But $q_U \subseteq Q_U$ so that $q_U=Q_U$ so that $q=Q$. \qed \\

\begin{cor}
If $q \neq q'$ satisfy $q \cap R = q' \cap R$, then $q,q'$ are incomparable; that is, neither $q \subseteq q'$ nor $q' \subseteq q$. 
\end{cor}

\begin{lem}[Going Up]
Let $R \subseteq S$ be an integral extension with $p \subsetneq P$ primes of $R$. Assume that $q$ is a prime of $S$ lying over $p$. Then there exists a $Q \in \spec S$ with $Q \supsetneq q$ and $Q \cap R=P$. 
\end{lem}

\noindent Proof: Set $U=R \setminus P$ and localize at $U$ (this is because we do not care about primes bigger than $P$ so we may as well blow them away). We know $R_U=R_P \subseteq S_U$ is an integral extension and $PR_P$ is the unique maximal ideal. We have $q_U \in \spec S_U$. Choose any maximal ideal $Q_U$ of $S_U$ containing $q_U$. Then $Q_U \cap R_U$ is maximal in $R_U$ so that $Q_U \cap R_U =P_U$. Hence $Q \cap R=P$ by the one-to-one correspondence for $\spec$ of localizations. We need $Q \supsetneq q$ and we have $Q \supseteq q$ by choice of $Q_U$. If $Q=q$ then $q \cap R=Q \cap R$ so that $P=p$, a contradiction. \qed \\

A natural question is if (faithfully) flat extensions have the Going Up property. The answer is no. Take $R=\Z$ and $S=\Z[x]$. This is faithfully flat. Take $q=(1+2x)S$ so that $q \cap \Z=(0)=P$ (as $q$ has no constants) and then $P=(2)R$. If going up held there would be a $Q \in \spec S$ with $Q \cap \Z=(2)$ and $1+2x \in Q$. If $Q \cap \Z=(2)$ then $2x \in Q$ then $2 \in Q$ so that $1 \in Q$ as $1=(1+2x)-2x$. But then $Q \cap \Z=\Z$, a contradiction. 

\begin{thmm}
Let $R \subseteq S$ be a ring extension satisfying Lying Over, Going Up, and Incomparable. Then $\dim R=\dim S$. 
\end{thmm}

\noindent Proof: We need show both inequalities. To show $\leq$, let $p_0 \subsetneq p_1 \subsetneq \cdots \subsetneq p_n$ be primes in $R$. By Lying Over, there is a $q_0 \in \spec S$ with $q_0 \cap R =p_0$. By Going Up, we get a chain of distinct primes $q_0 \subseteq q_1 \subseteq \cdots \subseteq q_n$ in $S$ so that $n \leq \dim S$. 

To see $\geq$, let $Q_0 \subseteq \cdots \subseteq Q_n$ be a chain of primes in $S$. When we collapse down to $R$, by Incomparable, nothing collapses so we get a chain $p_0 \subsetneq p_1 \subsetneq \cdots \subsetneq p_n$ with $Q_i \cap R=p_i$. \qed \\

\begin{ex}
Any ring of algebraic integers, i.e. $\Z$ in a number field, has dimension 1 since it is integral over $\Z$. Specific examples include $\Z[i]$ and $\Z\left[\frac{1+\sqrt{2}}{2}\right]$. 
\end{ex} 

\begin{ex}
We know that $k[s^9,s^6t^2,st^{103},t^{11111}]$ has dimension 2 since $k[s,t]$ is integral over it. 
\end{ex}

\begin{prop}
If $R \subseteq S$, where $S$ is noetherian, is an integral extension then $\phi^\#: \spec S \to \spec R$ is a finite mapping, i.e. finite to one. 
\end{prop}

\noindent Proof: We know by Lying Over that there is at least one $q$. Then $p \subseteq pS \cap R \subseteq q \cap R=p$ so that we must have all equalities which gives $p(S \cap R)=p$. We claim the $q$'s lying over $p$ are all minimal over $pS$. To see this, if $pS \subseteq q \subsetneq q$ and $q' \cap R=p$, then we would have $q \cap R=p$ as well. By Incomparable, we must have equality so that there are only finitely many $q$'s since the minimal primes of $pS$ is a finite set. \qed \\

Furthermore, Zariski's Main Theorem states that any ring homomorphism inducing a finite mapping on $\spec$ is a composition of localization and integral extensions. But this theorem is very deep and very hard. 

\begin{dfn}[Going Down]
A ring extension $R \subseteq S$ satisfies Going Down if given any primes $p \subsetneq P$ of $R$ and any $Q \in \spec S$ lying over $p$, there is a $q \subsetneq Q$ with $q \cap R=p$. 
\end{dfn}

\begin{rem}
Arbitrary integral extensions do not satisfy this. For example, take $R=k[s] \subseteq k[x,y]/(xy,y^2-y)$. Then this is an integral extension since it is generated by $y$ and $y$ is a root of $y^2-y$, i.e. $t^2-t \in R[t]$. But Going Down fails. Set $p=(0) \subsetneq P=(x)$ and $Q=(1-y)S$. Notice that $Q \supset (x)=P$ since $x=x(1-y)$ as $xy=0$. So $Q \cap R \supset P$ and $P$ is maximal so that $Q \cap R=P$. However, $Q$ is also a minimal prime ideal of $S$. If $q \subsetneq Q$ then $1-y \notin q$. It cannot contain $y$ since then $Q$ would contain 1. But it has to contain $y^2-y=y(1-y)$ so it contains one or the other. But then no such $q$ exists to satisfy Going Down. 
\end{rem}

\begin{rem}
Going Down still fails to hold in arbitrary integral extensions even if $R,S$ are assumed to be domains. You need more still.
\end{rem}

\begin{thmm}
Let $R \subseteq S$ be and integral extension, where $R,S$ are both integral domains with $R$ normal. Then $R \subseteq S$ satisfies Going Down.
\end{thmm}

\begin{prop}
Let $\phi: R \to S$ be a flat ring homomorphism. Then $\phi$ satisfies Going Down.
\end{prop}

\noindent Proof: Let $p \subsetneq P$ be primes in $R$ and $Q \in \spec S$ lying over $P$. Localize at $P,Q$: $\frac{\phi}{1}: R_p \to S_Q$ is a flat local ring homomorphism. Hence, $\frac{\phi}{1}$ is faithfully flat. So $\left(\frac{\phi}{1}\right)^\#: \spec S_Q \to \spec R_P$ is surjective. Then there is a $q_Q$ lying over $p_P$. Hence, $q$ lies over $p$. \qed \\

\begin{lem}
Let $(R,\fm)$ be a noetherian local ring. Then there is a system of parameters for $R$; that is, elements $x_1,\cdots,x_d \in \fm$ with $d=\dim R$ and $\sqrt{(x_1,\cdots,x_d)}=\fm$. 
\end{lem}

\noindent Proof: We proceed by induction on $d=\dim R$. If $d=0$ then $R$ is artinian so $\fm$ is nilpotent. So $\fm=\sqrt{(0)}$ is generated by 0 elements. Now assume that $d \geq 1$ and $p_1,\cdots,p_s$ are primes of $R$ such that $\dim R/p_i=\dim R$ (these are minimal primes that occur in chains of maximal length). Since $d \geq 1$ and $\dim R/\fm=0$, $\fm$ is not among the $p_i$'s. So $\fm \subsetneq \bigcup_{i=1}^s p_i$ by Prime Avoidance. Take $x_1 \in \fm$ so that $x_1 \notin p_i$ for all $i$. Then $\dim R/(x_1) < \dim R$ (we know that $\dim R/(x_1) \geq \dim R-1$ so it must be equal to it). Then it follows by induction that there are $x_2,\cdots,x_d$ such that $\sqrt{(\ov{x}_2,\cdots,\ov{x}_d)}= \ov{\fm}$ so $\sqrt{x_1,\cdots,x_d}=\fm$. \qed \\

\begin{lem}
Let $(R,\fm)$ be a noetherian local ring. If $\sqrt{(x_1,\cdots,x_t)}=\fm$, then $t \geq d$. 
\end{lem}

\noindent Proof: We know that the dimension drops by 1 each time and $\sqrt{(x_1,\cdots,x_n)}=\fm$ if and only if $\fm^n \subseteq (x_1,\cdots,x_n)$ for some $n$ if and only if $R/(x_1,\cdots,x_n)$ is artinian, i.e. 0-dimensional. \qed \\

\begin{thmm}
Let $\phi: R \to S$ be a homomorphism of noetherian rings and $q \in \spec S$ with $p=q \cap R$. Then
\begin{enumerate}[(i)]
\item $\htt q \leq \htt p + \dim(S_q/pS_q)$
\item $\htt q = \htt p + \dim(S_q/pS_q)$ if $\phi$ satisfies Going Down, e.g. if $\phi$ is flat. 
\end{enumerate}
In particular, if $\phi: (R,\fm) \to (S, \eta)$ is flat and local then $\dim S=\dim R+\dim(S/\fm S)$. 
\end{thmm}

\noindent Proof: Replace $R,S$ by $R_p,S_q$ to assume that $\phi: (R,\fm) \to (S,\eta)$ is a local ring homomorphism. We shall prove that $\dim S \leq \dim R + \dim (S/\fm S)$. Let $d=\dim R$ and $e=\dim (S/\fm S)$. Let $x_1,\cdots,x_d$ be a system of parameters for $R$ and $y_1,\cdots,y_e \in S$ be such that their images in $S/\fm S$ form a system of parameters. Then $\fm^N \subseteq (x_1,\cdots,x_d)$ for some $N$ and $\eta^{N'} \subseteq (y_1,\cdots,y_e)+\fm S$ for some $N'$. But then 
\[
\begin{split}
\eta^{N'N} &\subseteq \big((y_1,\cdots,y_e)+\fm S\big)^N \\
&\subseteq (y_1,\cdots,y_e)+\fm^NS \\
&\subseteq (y_1,\cdots,y_e) + (x_1,\cdots,x_d)S
\end{split}
\]
so $\dim S \leq d+e$ since $x_1,\cdots,x_d,y_1,\cdots,y_e$ are a system of parameters for $S$.

For the second part, assume that $\phi$ has Going Down. Let $\ov{Q}_0 \subsetneq \ov{Q}_1 \subseteq \cdots \subsetneq \ov{Q}_e$ be primes in $S/\fm S$ and lift to a chain in $S$: $\fm S \subseteq Q_0 \subsetneq Q_1 \subsetneq \cdots \subsetneq Q_e=\eta$. Then $Q_i \cap R=\fm$ for $i=0,1,\cdots,e$ as they contain $\fm$ but $\fm$ is maximal. So take a maximal chain of primes in $R$ $p_0 \subsetneq p_1 \subsetneq \cdots \subsetneq p_d=\fm$. There exists $q_{d-1},\cdots,q_0$ primes of $S$ with $q_0 \subsetneq \cdots \subsetneq q_{d-1} \subsetneq Q_0$ such that $q_i \cap R=p_i$ for $i=0,1,\cdots,d$. Therefore, $\dim S \geq d+e$. \qed \\

\begin{cor}
If $(R,\fm)$ is a noetherian local ring, then $\dim \hat{R}=\dim R$.
\end{cor}

\noindent Proof: If $R \to \hat{R}$ is flat, then $\dim \hat{R}=\dim R+\dim(\hat{R}/\fm \hat{R})$, the right part if $R/\fm$ is a field do that its dimension is 0. \qed \\